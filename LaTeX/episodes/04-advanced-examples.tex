\section[Avanzadas]{Episodio 4: Simulaciones avanzadas y energ\'ia libre}

\subsection{Objetivos y mapa}

\begin{frame}
\frametitle{Objetivos del episodio}
\begin{itemize}
  \item Aplicar t\'ecnicas avanzadas de muestreo y energ\'ia libre.
  \item Dise\~nar protocolos robustos para sistemas complejos.
  \item Evaluar convergencia y reproducibilidad.
\end{itemize}
\end{frame}

\begin{frame}
\frametitle{Mapa de t\'ecnicas avanzadas}
\begin{itemize}
  \item Termostatos y barostatos de alta estabilidad.
  \item Integraci\'on multiescala y restricciones.
  \item Muestreo mejorado: umbrella, metadin\'amica, REMD.
  \item Energ\'ias libres: FEP, TI, BAR/MBAR.
\end{itemize}
\end{frame}

\begin{frame}
\frametitle{Variables de control}
\begin{align*}
  \beta &= \frac{1}{\kbt}, & \lambda &\in [0,1].
\end{align*}
\begin{itemize}
  \item La variable \(\lambda\) conecta estados inicial y final.
  \item \(\beta\) define el peso estad\'istico en el ensamble.
\end{itemize}
\end{frame}

\begin{frame}
\frametitle{Requisitos de datos}
\begin{itemize}
  \item Trayectorias suficientemente largas para cada ventana.
  \item Muestreo independiente entre ventanas o replicas.
  \item Reportes de energ\'ias y variables colectivas.
\end{itemize}
\end{frame}

\begin{frame}
\frametitle{Flujo general}
\begin{enumerate}
  \item Preparaci\'on y equilibrado estricto.
  \item Definici\'on de CVs y ventanas.
  \item Producci\'on y chequeo de convergencia.
  \item Postproceso con estimadores robustos.
\end{enumerate}
\end{frame}

\subsection{Control termodin\'amico}

\begin{frame}
\frametitle{Ensembles avanzados}
\begin{itemize}
  \item NVT para control de temperatura.
  \item NPT para presi\'on y densidad realistas.
  \item NP\(\gamma\)T para tensi\'on superficial en membranas.
\end{itemize}
\end{frame}

\begin{frame}
\frametitle{Ecuaci\'on de Langevin}
\begin{align*}
  m\ddot{\mathbf{r}} &= -\nabla U(\mathbf{r}) - \gamma m \dot{\mathbf{r}} + \sqrt{2\gamma m \kbt}\, \mathbf{R}(t).
\end{align*}
\begin{itemize}
  \item Controla la temperatura con fricci\'on y ruido.
\end{itemize}
\end{frame}

\begin{frame}
\frametitle{Nos\'e-Hoover en cadena}
\begin{align*}
  \dot{\eta} &= \frac{1}{Q}\left(\sum_i \frac{p_i^2}{m_i} - N_f \kbt\right).
\end{align*}
\begin{itemize}
  \item Permite muestreo can\'onico con menos ruido.
\end{itemize}
\end{frame}

\begin{frame}
\frametitle{Equipartici\'on de energ\'ia}
\begin{align*}
  \langle K \rangle &= \frac{N_f}{2} \kbt.
\end{align*}
\begin{itemize}
  \item Verificar temperaturas por subgrupos (agua, soluto).
\end{itemize}
\end{frame}

\begin{frame}
\frametitle{Diagn\'ostico de temperatura}
\begin{itemize}
  \item Histograma de velocidades vs Maxwell-Boltzmann.
  \item Monitor de deriva en sistemas no equilibrados.
  \item Control de saltos al cambiar par\'ametros.
\end{itemize}
\end{frame}

\subsection{Presi\'on y barostatos}

\begin{frame}
\frametitle{Tipos de barostato}
\begin{itemize}
  \item Isotr\'opico: caja escala igual en todas las direcciones.
  \item Anisotr\'opico: permite cambios independientes.
  \item Barostato Monte Carlo: cambios discretos en volumen.
\end{itemize}
\end{frame}

\begin{frame}
\frametitle{Fluctuaciones de volumen}
\begin{align*}
  \kappa_T &= \frac{\langle V^2 \rangle - \langle V \rangle^2}{\kbt\,\langle V \rangle}.
\end{align*}
\begin{itemize}
  \item Compara con valores experimentales si existen.
\end{itemize}
\end{frame}

\begin{frame}
\frametitle{Compresibilidad y estabilidad}
\begin{itemize}
  \item Valores demasiado altos indican mala convergencia.
  \item Ajustar frecuencia de barostato y amortiguamiento.
\end{itemize}
\end{frame}

\begin{frame}
\frametitle{Tensi\'on superficial}
\begin{align*}
  \gamma &= \frac{L_z}{2}\left(P_{zz} - \frac{P_{xx}+P_{yy}}{2}\right).
\end{align*}
\begin{itemize}
  \item Relevante en membranas y bicapas lip\'idicas.
\end{itemize}
\end{frame}

\begin{frame}
\frametitle{Control de presi\'on}
\begin{itemize}
  \item Registrar volumen y densidad por ventana.
  \item Evitar acoplamientos demasiado agresivos.
  \item Estabilizar antes de muestreos largos.
\end{itemize}
\end{frame}

\subsection{Restricciones e integraci\'on}

\begin{frame}
\frametitle{Restricciones geom\'etricas}
\begin{itemize}
  \item SHAKE/LINCS fijan enlaces r\'apidos.
  \item Permiten pasos de integraci\'on m\'as grandes.
\end{itemize}
\end{frame}

\begin{frame}
\frametitle{Integrador Verlet}
\begin{align*}
  \mathbf{r}_{t+\Delta t} &= \mathbf{r}_t + \mathbf{v}_t\Delta t + \frac{1}{2}\mathbf{a}_t\Delta t^2.
\end{align*}
\begin{itemize}
  \item Conserva energ\'ia en integraciones cortas.
\end{itemize}
\end{frame}

\begin{frame}
\frametitle{Separaci\'on de escalas (RESPA)}
\begin{align*}
  U &= U_{\text{r\'apido}} + U_{\text{lento}}.
\end{align*}
\begin{itemize}
  \item Integra fuerzas lentas con pasos m\'as grandes.
\end{itemize}
\end{frame}

\begin{frame}
\frametitle{Reparto de masas}
\begin{itemize}
  \item HMR: aumenta masa de H para estabilidad.
  \item Permite \(\Delta t\) de 4--5 fs en sistemas bien restringidos.
\end{itemize}
\end{frame}

\begin{frame}
\frametitle{Estabilidad num\'erica}
\begin{itemize}
  \item Monitor de deriva de energ\'ia total.
  \item Ajustar \(\Delta t\) seg\'un rigidez y temperatura.
\end{itemize}
\end{frame}

\subsection{Sistemas complejos}

\begin{frame}
\frametitle{Membranas y sistemas anisotr\'opicos}
\begin{itemize}
  \item Ajuste de presi\'on independiente por eje.
  \item Control de grosor y \'area por l\'ipido.
\end{itemize}
\end{frame}

\begin{frame}
\frametitle{Bicapa lip\'idica}
\begin{center}
  \includegraphics[width=0.78\linewidth]{episodes/lipids_bilayer.pdf}
\end{center}
\end{frame}

\begin{frame}
\frametitle{Solvataci\'on expl\'icita vs impl\'icita}
\begin{align*}
  \Delta G_{\text{solv}} &= \Delta G_{\text{elec}} + \Delta G_{\text{np}}.
\end{align*}
\begin{itemize}
  \item Modelos impl\'icitos reducen el coste computacional.
\end{itemize}
\end{frame}

\begin{frame}
\frametitle{Pantalla i\'onica}
\begin{align*}
  \kappa^{-1} &= \sqrt{\frac{\varepsilon \kbt}{2 N_A e^2 I}}.
\end{align*}
\begin{itemize}
  \item La fuerza i\'onica \(I\) controla el apantallamiento.
\end{itemize}
\end{frame}

\begin{frame}
\frametitle{Energ\'ia de desolvataci\'on}
\begin{center}
  \includegraphics[width=0.7\linewidth]{episodes/gsd_desolvation.png}
\end{center}
\end{frame}

\subsection{Muestreo mejorado}

\begin{frame}
\frametitle{Barreras y eventos raros}
\begin{itemize}
  \item Transiciones con \(\Delta G^{\ddagger}\) altas son infrecuentes.
  \item Se requiere un potencial de sesgo para acelerar el muestreo.
\end{itemize}
\end{frame}

\begin{frame}
\frametitle{Muestreo con ventanas (umbrella)}
\begin{align*}
  U_{\text{sesgo}}(\xi) &= \frac{k}{2}(\xi-\xi_0)^2.
\end{align*}
\begin{itemize}
  \item Controla el muestreo en ventanas de la CV \(\xi\).
\end{itemize}
\end{frame}

\begin{frame}
\frametitle{PMF desde distribuciones}
\begin{align*}
  F(\xi) &= -\kbt \ln P(\xi) + C.
\end{align*}
\begin{itemize}
  \item WHAM combina hist\'ogramas de varias ventanas.
\end{itemize}
\end{frame}

\begin{frame}
\frametitle{Intercambio de r\'eplicas}
\begin{align*}
  P_{\text{acc}} &= \min\left(1, \exp\left[(\beta_i-\beta_j)(U_j-U_i)\right]\right).
\end{align*}
\begin{itemize}
  \item Intercambio de temperaturas para superar barreras.
\end{itemize}
\end{frame}

\begin{frame}
\frametitle{Muestreo adaptativo}
\begin{itemize}
  \item Selecci\'on de nuevas semillas por incertidumbre.
  \item Combina simulaciones cortas con an\'alisis iterativo.
\end{itemize}
\end{frame}

\subsection{Metadin\'amica}

\begin{frame}
\frametitle{Variables colectivas}
\begin{itemize}
  \item Coordenadas reducidas que describen la transici\'on.
  \item Deben separar estados metaestables.
\end{itemize}
\end{frame}

\begin{frame}
\frametitle{Bias por deposici\'on de gaussianas}
\begin{align*}
  V(s,t) &= \sum_{i} w\exp\left[-\frac{(s-s_i)^2}{2\sigma^2}\right].
\end{align*}
\end{frame}

\begin{frame}
\frametitle{Metadin\'amica temperada}
\begin{align*}
  w(t) &= w_0\exp\left(-\frac{V(s,t)}{\kbt\,\Delta T}\right).
\end{align*}
\begin{itemize}
  \item Suaviza el crecimiento del sesgo en el tiempo.
\end{itemize}
\end{frame}

\begin{frame}
\frametitle{Reconstrucci\'on de energ\'ia libre}
\begin{align*}
  F(s) &\approx -\frac{T+\Delta T}{\Delta T} V(s,t\to\infty).
\end{align*}
\end{frame}

\begin{frame}
\frametitle{Riesgos comunes}
\begin{itemize}
  \item CVs mal elegidas generan artefactos.
  \item Intervalo de deposici\'on demasiado corto.
  \item Falta de muestreo en regiones clave.
\end{itemize}
\end{frame}

\subsection{FEP y TI}

\begin{frame}
\frametitle{Perturbaci\'on de energ\'ia libre}
\begin{align*}
  \Delta G &= -\kbt \ln \left\langle \exp\left[-\beta\Delta U\right]\right\rangle_0.
\end{align*}
\end{frame}

\begin{frame}
\frametitle{Integraci\'on termodin\'amica}
\begin{align*}
  \Delta G &= \int_0^1 \left\langle \frac{\partial U(\lambda)}{\partial \lambda} \right\rangle_{\lambda} d\lambda.
\end{align*}
\end{frame}

\begin{frame}
\frametitle{Esquema FEP}
\begin{center}
  \includegraphics[width=0.7\linewidth]{fep.png}
\end{center}
\end{frame}

\begin{frame}
\frametitle{Potenciales suavizados (soft-core)}
\begin{itemize}
  \item Evitan singularidades al apagar interacciones.
  \item Aseguran solapamiento de distribuciones vecinas.
\end{itemize}
\end{frame}

\begin{frame}
\frametitle{Convergencia de \(\lambda\)}
\begin{itemize}
  \item Aumentar densidad de ventanas en cambios r\'apidos.
  \item Monitor de hist\'ogramas de \(\partial U/\partial \lambda\).
\end{itemize}
\end{frame}

\subsection{Ciclos termodin\'amicos}

\begin{frame}
\frametitle{Doble desacoplamiento}
\begin{itemize}
  \item Se apagan interacciones del ligando en solvente y complejo.
  \item La diferencia proporciona \(\Delta G\) de uni\'on.
\end{itemize}
\end{frame}

\begin{frame}
\frametitle{Ciclo termodin\'amico}
\begin{center}
  \includegraphics[width=0.72\linewidth]{episodes/thermodynamiccycle.pdf}
\end{center}
\end{frame}

\begin{frame}
\frametitle{Relaci\'on con afinidad}
\begin{align*}
  \Delta G_{\text{bind}} &= \kbt \ln K_d.
\end{align*}
\begin{itemize}
  \item Usar correcciones de estado est\'andar si procede.
\end{itemize}
\end{frame}

\begin{frame}
\frametitle{Cierre del ciclo}
\begin{align*}
  \sum_{\text{ciclo}} \Delta G_i &= 0.
\end{align*}
\begin{itemize}
  \item Una desviaci\'on indica errores sistem\'aticos.
\end{itemize}
\end{frame}

\begin{frame}
\frametitle{Propagaci\'on de errores}
\begin{itemize}
  \item Errores de ventanas se suman en cuadratura.
  \item Reportar intervalos de confianza.
\end{itemize}
\end{frame}

\subsection{No equilibrio}

\begin{frame}
\frametitle{Igualdad de Jarzynski}
\begin{align*}
  \Delta G &= -\kbt \ln \langle e^{-\beta W} \rangle.
\end{align*}
\end{frame}

\begin{frame}
\frametitle{Relaci\'on de Crooks}
\begin{align*}
  \frac{P_F(W)}{P_R(-W)} &= e^{\beta(W-\Delta G)}.
\end{align*}
\end{frame}

\begin{frame}
\frametitle{Steered MD}
\begin{align*}
  W &= \int_0^{\tau} \mathbf{F}_{\text{pull}}(t)\cdot d\mathbf{x}.
\end{align*}
\begin{itemize}
  \item Trabajo depende de la velocidad de tirado.
\end{itemize}
\end{frame}

\begin{frame}
\frametitle{Coordenada de reacci\'on}
\begin{center}
  \includegraphics[width=0.7\linewidth]{episodes/gs_ts.png}
\end{center}
\end{frame}

\subsection{Sampling avanzado oficial}

\begin{frame}
\frametitle{Replica Exchange Solute Tempering (REST)}
\begin{itemize}
  \item REST usa `OpenMMTools` para escalar energías en un subconjunto de átomos (OpenMM Cookbook REST).
  \item Se definien `CustomBondForce`, `CustomAngleForce`, `CustomTorsionForce` y `NonbondedForce` modificados, y se intercambian réplicas con `ReplicaExchangeSampler`.
  \item Sirve para el complejo proteína-ligando, donde solo el ligando se calienta manteniendo el entorno estable.
\end{itemize}
\end{frame}

\begin{frame}
\frametitle{Umbrella Sampling guiado}
\begin{itemize}
  \item Se aplica un `HarmonicBias` sobre una variable colectiva $x(r)$, y se recolectan histogramas por ventana (OpenMM Cookbook Umbrella Sampling).
  \item A partir de \(\Delta F(x) = -\kbt \log P(x)\), se reconstruye el perfil de energía libre y se identifican barreras.
  \item Combine con `argón-chemical-potential.py` para validar sobre potenciales LJ simples antes de usarlo en proteínas reales.
\end{itemize}
\end{frame}

\begin{frame}
\frametitle{Velocidad de tirado}
\begin{itemize}
  \item Tirados lentos aproximan cuasi-equilibrio.
  \item Tirados r\'apidos requieren muchas repeticiones.
\end{itemize}
\end{frame}

\subsection{Convergencia y estad\'istica}

\begin{frame}
\frametitle{Promedios por bloques}
\begin{itemize}
  \item Dividir la serie temporal en bloques iguales.
  \item Estimar media y error con bloques independientes.
\end{itemize}
\end{frame}

\begin{frame}
\frametitle{Ineficiencia estad\'istica}
\begin{align*}
  g &= 1 + 2\sum_{t=1}^{\infty} \rho(t).
\end{align*}
\begin{itemize}
  \item Reduce el n\'umero efectivo de muestras.
\end{itemize}
\end{frame}

\begin{frame}
\frametitle{BAR/MBAR}
\begin{itemize}
  \item BAR minimiza varianza entre dos estados.
  \item MBAR generaliza a m\'ultiples ventanas.
\end{itemize}
\end{frame}

\begin{frame}
\frametitle{Perfil de energ\'ia libre}
\begin{center}
  \includegraphics[width=0.72\linewidth]{episodes/betacorrectedfel.pdf}
\end{center}
\end{frame}

\begin{frame}
\frametitle{Lista de verificaci\'on de convergencia}
\begin{itemize}
  \item Hist\'ogramas solapados entre ventanas.
  \item Errores estables en el tiempo.
  \item Repeticiones con resultados consistentes.
\end{itemize}
\end{frame}

\subsection{Resumen}

\begin{frame}
\frametitle{Resumen del episodio}
\begin{itemize}
  \item Se revisaron protocolos avanzados y energ\'ias libres.
  \item Se enfatiz\'o la convergencia y el control estad\'istico.
\end{itemize}
\end{frame}

\begin{frame}
\frametitle{Buenas pr\'acticas}
\begin{itemize}
  \item Documentar versiones, semillas y par\'ametros.
  \item Automatizar an\'alisis y controles.
\end{itemize}
\end{frame}

\begin{frame}
\frametitle{Errores frecuentes}
\begin{itemize}
  \item Ventanas insuficientes en cambios r\'apidos.
  \item Falta de solapamiento entre \(\lambda\).
\end{itemize}
\end{frame}

\begin{frame}
\frametitle{Reproducibilidad}
\begin{itemize}
  \item Guardar scripts y semillado.
  \item Reportar incertidumbre y condiciones iniciales.
\end{itemize}
\end{frame}

\begin{frame}
\frametitle{Siguiente paso}
\begin{itemize}
  \item Entrar en el an\'alisis detallado de trayectorias.
\end{itemize}
\end{frame}
