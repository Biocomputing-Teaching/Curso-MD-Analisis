\section[Advanced]{Episode 4: Advanced simulations and free energy}

\subsection{Objectives and roadmap}

\begin{frame}
\frametitle{Episode objectives}
\begin{itemize}
  \item Apply advanced sampling and free energy techniques.
  \item Design robust protocols for complex systems.
  \item Evaluate convergence and reproducibility.
\end{itemize}
\end{frame}

\begin{frame}
\frametitle{Advanced techniques map}
\begin{itemize}
  \item Highly stable thermostats and barostats.
  \item Multiscale integration and constraints.
  \item Enhanced sampling: umbrella, metadynamics, REMD.
  \item Free energies: FEP, TI, BAR/MBAR.
\end{itemize}
\end{frame}

\begin{frame}
\frametitle{Control variables}
\begin{align*}
  \beta &= \frac{1}{\kbt}, & \lambda &\in [0,1].
\end{align*}
\begin{itemize}
  \item The variable \(\lambda\) bridges the initial and final states.
  \item \(\beta\) defines the statistical weight in the ensemble.
\end{itemize}
\end{frame}

\begin{frame}
\frametitle{Data requirements}
\begin{itemize}
  \item Trajectories long enough for each window.
  \item Independent sampling across windows or replicas.
  \item Energy reports and collective variables.
\end{itemize}
\end{frame}

\begin{frame}
\frametitle{General workflow}
\begin{enumerate}
  \item Strict preparation and equilibration.
  \item Define CVs and windows.
  \item Production and convergence checks.
  \item Post-process with robust estimators.
\end{enumerate}
\end{frame}

\subsection{Thermodynamic control}

\begin{frame}
\frametitle{Advanced ensembles}
\begin{itemize}
  \item NVT for temperature control.
  \item NPT for realistic pressure and density.
  \item NP\(\gamma\)T for surface tension in membranes.
\end{itemize}
\end{frame}

\begin{frame}
\frametitle{Langevin equation}
\begin{align*}
  m\ddot{\mathbf{r}} &= -\nabla U(\mathbf{r}) - \gamma m \dot{\mathbf{r}} + \sqrt{2\gamma m \kbt}\, \mathbf{R}(t).
\end{align*}
\begin{itemize}
  \item Controls temperature via friction and noise.
\end{itemize}
\end{frame}

\begin{frame}
\frametitle{Nos\'e-Hoover chain}
\begin{align*}
  \dot{\eta} &= \frac{1}{Q}\left(\sum_i \frac{p_i^2}{m_i} - N_f \kbt\right).
\end{align*}
\begin{itemize}
  \item Allows canonical sampling with less noise.
\end{itemize}
\end{frame}

\begin{frame}
\frametitle{Energy equipartition}
\begin{align*}
  \langle K \rangle &= \frac{N_f}{2} \kbt.
\end{align*}
\begin{itemize}
  \item Check temperatures per subgroup (water, solute).
\end{itemize}
\end{frame}

\begin{frame}
\frametitle{Temperature diagnostics}
\begin{itemize}
  \item Speed histogram vs Maxwell-Boltzmann.
  \item Drift monitor for non-equilibrated systems.
  \item Watch for jumps when parameters change.
\end{itemize}
\end{frame}

\subsection{Pressure and barostats}

\begin{frame}
\frametitle{Types of barostat}
\begin{itemize}
  \item Isotropic: box scales equally in all directions.
  \item Anisotropic: allows independent scaling.
  \item Monte Carlo barostat: discrete volume changes.
\end{itemize}
\end{frame}

\begin{frame}
\frametitle{Volume fluctuations}
\begin{align*}
  \kappa_T &= \frac{\langle V^2 \rangle - \langle V \rangle^2}{\kbt\,\langle V \rangle}.
\end{align*}
\begin{itemize}
  \item Compare with experimental values if available.
\end{itemize}
\end{frame}

\begin{frame}
\frametitle{Compressibility and stability}
\begin{itemize}
  \item Excessive values signal poor convergence.
  \item Adjust barostat frequency and damping.
\end{itemize}
\end{frame}

\begin{frame}
\frametitle{Surface tension}
\begin{align*}
  \gamma &= \frac{L_z}{2}\left(P_{zz} - \frac{P_{xx}+P_{yy}}{2}\right).
\end{align*}
\begin{itemize}
  \item Relevant for membranes and lipid bilayers.
\end{itemize}
\end{frame}

\begin{frame}
\frametitle{Pressure control}
\begin{itemize}
  \item Log volume and density per window.
  \item Avoid overly aggressive couplings.
  \item Stabilize before long sampling runs.
\end{itemize}
\end{frame}

\subsection{Constraints and integration}

\begin{frame}
\frametitle{Geometric constraints}
\begin{itemize}
  \item SHAKE/LINCS fix fast bonds.
  \item Allow larger integration steps.
\end{itemize}
\end{frame}

\begin{frame}
\frametitle{Verlet integrator}
\begin{align*}
  \mathbf{r}_{t+\Delta t} &= \mathbf{r}_t + \mathbf{v}_t\Delta t + \frac{1}{2}\mathbf{a}_t\Delta t^2.
\end{align*}
\begin{itemize}
  \item Conserves energy in short integrations.
\end{itemize}
\end{frame}

\begin{frame}
\frametitle{Scale separation (RESPA)}
\begin{align*}
  U &= U_{\text{fast}} + U_{\text{slow}}.
\end{align*}
\begin{itemize}
  \item Integrate slow forces with larger steps.
\end{itemize}
\end{frame}

\begin{frame}
 \frametitle{Mass repartitioning}
\begin{itemize}
  \item HMR: increases hydrogen mass for stability.
  \item Enables \(\Delta t\) of 4--5 fs in well-constrained systems.
\end{itemize}
\end{frame}

\begin{frame}
\frametitle{Numerical stability}
\begin{itemize}
  \item Monitor total energy drift.
  \item Adjust \(\Delta t\) according to stiffness and temperature.
\end{itemize}
\end{frame}

\subsection{Complex systems}

\begin{frame}
\frametitle{Membranes and anisotropic systems}
\begin{itemize}
  \item Independent pressure scaling per axis.
  \item Control thickness and area per lipid.
\end{itemize}
\end{frame}

\begin{frame}
\frametitle{Lipid bilayer}
\begin{center}
  \includegraphics[width=0.78\linewidth]{episodes/lipids_bilayer.pdf}
\end{center}
\end{frame}

\begin{frame}
\frametitle{Explicit vs implicit solvation}
\begin{align*}
  \Delta G_{\text{solv}} &= \Delta G_{\text{elec}} + \Delta G_{\text{np}}.
\end{align*}
\begin{itemize}
  \item Implicit models reduce computational cost.
\end{itemize}
\end{frame}

\begin{frame}
\frametitle{Ionic screening}
\begin{align*}
  \kappa^{-1} &= \sqrt{\frac{\varepsilon \kbt}{2 N_A e^2 I}}.
\end{align*}
\begin{itemize}
  \item The ionic strength \(I\) controls the screening.
\end{itemize}
\end{frame}

\begin{frame}
\frametitle{Desolvation energy}
\begin{center}
  \includegraphics[width=0.7\linewidth]{episodes/gsd_desolvation.png}
\end{center}
\end{frame}

\subsection{Enhanced sampling}

\begin{frame}
\frametitle{Barriers and rare events}
\begin{itemize}
  \item Transitions with high \(\Delta G^{\ddagger}\) are rare.
  \item A bias potential accelerates sampling.
\end{itemize}
\end{frame}

\begin{frame}
\frametitle{Window sampling (umbrella)}
\begin{align*}
  U_{\text{sesgo}}(\xi) &= \frac{k}{2}(\xi-\xi_0)^2.
\end{align*}
\begin{itemize}
  \item Controls sampling in windows of the CV \(\xi\).
\end{itemize}
\end{frame}

\begin{frame}
\frametitle{PMF from distributions}
\begin{align*}
  F(\xi) &= -\kbt \ln P(\xi) + C.
\end{align*}
\begin{itemize}
  \item WHAM combines histograms from multiple windows.
\end{itemize}
\end{frame}

\begin{frame}
\frametitle{Replica exchange}
\begin{align*}
  P_{\text{acc}} &= \min\left(1, \exp\left[(\beta_i-\beta_j)(U_j-U_i)\right]\right).
\end{align*}
\begin{itemize}
  \item Temperature swaps to overcome barriers.
\end{itemize}
\end{frame}

\begin{frame}
\frametitle{Adaptive sampling}
\begin{itemize}
  \item Select new seeds based on uncertainty.
  \item Combine short simulations with iterative analysis.
\end{itemize}
\end{frame}

\subsection{Metadynamics}

\begin{frame}
\frametitle{Collective variables}
\begin{itemize}
  \item Reduced coordinates that describe the transition.
  \item Must separate metastable states.
\end{itemize}
\end{frame}

\begin{frame}
\frametitle{Bias via Gaussian hills}
\begin{align*}
  V(s,t) &= \sum_{i} w\exp\left[-\frac{(s-s_i)^2}{2\sigma^2}\right].
\end{align*}
\end{frame}

\begin{frame}
\frametitle{Well-tempered metadynamics}
\begin{align*}
  w(t) &= w_0\exp\left(-\frac{V(s,t)}{\kbt\,\Delta T}\right).
\end{align*}
\begin{itemize}
  \item Smooths the growth of the bias over time.
\end{itemize}
\end{frame}

\begin{frame}
\frametitle{Free energy reconstruction}
\begin{align*}
  F(s) &\approx -\frac{T+\Delta T}{\Delta T} V(s,t\to\infty).
\end{align*}
\end{frame}

\begin{frame}
\frametitle{Common pitfalls}
\begin{itemize}
  \item Poorly chosen CVs create artifacts.
  \item Deposition interval too short.
  \item Lack of sampling in key regions.
\end{itemize}
\end{frame}

\subsection{FEP and TI}

\begin{frame}
\frametitle{Free energy perturbation}
\begin{align*}
  \Delta G &= -\kbt \ln \left\langle \exp\left[-\beta\Delta U\right]\right\rangle_0.
\end{align*}
\end{frame}

\begin{frame}
\frametitle{Thermodynamic integration}
\begin{align*}
  \Delta G &= \int_0^1 \left\langle \frac{\partial U(\lambda)}{\partial \lambda} \right\rangle_{\lambda} d\lambda.
\end{align*}
\end{frame}

\begin{frame}
\frametitle{FEP scheme}
\begin{center}
  \includegraphics[width=0.7\linewidth]{fep.png}
\end{center}
\end{frame}

\begin{frame}
\frametitle{Soft-core potentials}
\begin{itemize}
  \item Prevent singularities when turning off interactions.
  \item Ensure overlap of neighboring distributions.
\end{itemize}
\end{frame}

\begin{frame}
\frametitle{\(\lambda\) convergence}
\begin{itemize}
  \item Increase window density where changes are rapid.
  \item Monitor histograms of \(\partial U/\partial \lambda\).
\end{itemize}
\end{frame}

\subsection{Thermodynamic cycles}

\begin{frame}
\frametitle{Double decoupling}
\begin{itemize}
  \item Turn off ligand interactions in solvent and complex.
  \item The difference yields the binding \(\Delta G\).
\end{itemize}
\end{frame}

\begin{frame}
\frametitle{Thermodynamic cycle}
\begin{center}
  \includegraphics[width=0.72\linewidth]{episodes/thermodynamiccycle.pdf}
\end{center}
\end{frame}

\begin{frame}
\frametitle{Relation to affinity}
\begin{align*}
  \Delta G_{\text{bind}} &= \kbt \ln K_d.
\end{align*}
\begin{itemize}
  \item Apply standard-state corrections when appropriate.
\end{itemize}
\end{frame}

\begin{frame}
\frametitle{Cycle closure}
\begin{align*}
  \sum_{\text{ciclo}} \Delta G_i &= 0.
\end{align*}
\begin{itemize}
  \item A deviation indicates systematic errors.
\end{itemize}
\end{frame}

\begin{frame}
\frametitle{Error propagation}
\begin{itemize}
  \item Window errors add in quadrature.
  \item Report confidence intervals.
\end{itemize}
\end{frame}

\subsection{Non-equilibrium}

\begin{frame}
\frametitle{Jarzynski equality}
\begin{align*}
  \Delta G &= -\kbt \ln \langle e^{-\beta W} \rangle.
\end{align*}
\end{frame}

\begin{frame}
\frametitle{Crooks relation}
\begin{align*}
  \frac{P_F(W)}{P_R(-W)} &= e^{\beta(W-\Delta G)}.
\end{align*}
\end{frame}

\begin{frame}
\frametitle{Steered MD}
\begin{align*}
  W &= \int_0^{\tau} \mathbf{F}_{\text{pull}}(t)\cdot d\mathbf{x}.
\end{align*}
\begin{itemize}
  \item Work depends on the pulling speed.
\end{itemize}
\end{frame}

\begin{frame}
\frametitle{Reaction coordinate}
\begin{center}
  \includegraphics[width=0.7\linewidth]{episodes/gs_ts.png}
\end{center}
\end{frame}

\subsection{Official advanced sampling}

\begin{frame}
\frametitle{Replica Exchange Solute Tempering (REST)}
\begin{itemize}
  \item REST uses `OpenMMTools` to scale energies on a subset of atoms (OpenMM Cookbook REST).
  \item Define modified `CustomBondForce`, `CustomAngleForce`, `CustomTorsionForce`, and `NonbondedForce`, and swap replicas with `ReplicaExchangeSampler`.
  \item Useful for the protein–ligand complex where only the ligand is heated while the environment stays stable.
\end{itemize}
\end{frame}

\begin{frame}
\frametitle{Guided umbrella sampling}
\begin{itemize}
  \item Apply a `HarmonicBias` on a collective variable $x(r)$ and collect histograms per window (OpenMM Cookbook Umbrella Sampling).
  \item From \(\Delta F(x) = -\kbt \log P(x)\), reconstruct the free energy profile and identify barriers.
  \item Combine with \href{https://github.com/openmm/openmm/blob/master/examples/python-examples/argon-chemical-potential.py}{\texttt{argon-chemical-potential.py}} to validate over simple LJ potentials before applying it to real proteins.
\end{itemize}
\end{frame}

\begin{frame}
\frametitle{Pulling speed}
\begin{itemize}
  \item Slow pulls approximate quasi-equilibrium.
  \item Fast pulls require many repeats.
\end{itemize}
\end{frame}

\subsection{Convergence and statistics}

\begin{frame}
\frametitle{Block averages}
\begin{itemize}
  \item Split the time series into equal blocks.
  \item Estimate mean and error from independent blocks.
\end{itemize}
\end{frame}

\begin{frame}
\frametitle{Statistical inefficiency}
\begin{align*}
  g &= 1 + 2\sum_{t=1}^{\infty} \rho(t).
\end{align*}
\begin{itemize}
  \item Reduces the effective number of samples.
\end{itemize}
\end{frame}

\begin{frame}
\frametitle{BAR/MBAR}
\begin{itemize}
  \item BAR minimizes variance between two states.
  \item MBAR generalizes to multiple windows.
\end{itemize}
\end{frame}

\begin{frame}
\frametitle{Free energy profile}
\begin{center}
  \includegraphics[width=0.72\linewidth]{episodes/betacorrectedfel.pdf}
\end{center}
\end{frame}

\begin{frame}
\frametitle{Convergence checklist}
\begin{itemize}
  \item Overlapping histograms between windows.
  \item Stable errors over time.
  \item Repeats yielding consistent results.
\end{itemize}
\end{frame}

\subsection{Summary}

\begin{frame}
\frametitle{Episode summary}
\begin{itemize}
  \item Reviewed advanced protocols and free energies.
  \item Emphasized convergence and statistical control.
\end{itemize}
\end{frame}

\begin{frame}
\frametitle{Best practices}
\begin{itemize}
  \item Document versions, seeds, and parameters.
  \item Automate analyses and checks.
\end{itemize}
\end{frame}

\begin{frame}
\frametitle{Common mistakes}
\begin{itemize}
  \item Insufficient windows during rapid changes.
  \item Lack of overlap between \(\lambda\).
\end{itemize}
\end{frame}

\begin{frame}
\frametitle{Reproducibility}
\begin{itemize}
  \item Save scripts and seeding.
  \item Report uncertainty and initial conditions.
\end{itemize}
\end{frame}

\begin{frame}
\frametitle{Next step}
\begin{itemize}
  \item Dive into detailed trajectory analysis.
\end{itemize}
\end{frame}
