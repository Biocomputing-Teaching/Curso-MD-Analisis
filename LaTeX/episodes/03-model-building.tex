\section[Preparation]{Episode 3: System preparation and editing}

\subsection{Objectives and workflow}

\begin{frame}
\frametitle{Episode objectives}
\begin{itemize}
  \item Prepare systems ready for simulation in OpenMM.
  \item Standardize topologies and coordinates.
  \item Produce reproducible, verifiable inputs.
\end{itemize}
\end{frame}

\begin{frame}
\frametitle{Preparation workflow}
\begin{enumerate}
  \item Read the structure (PDB/MOL2/SDF).
  \item Repair and complete missing residues.
  \item Protonate and assign charges.
  \item Solvate, add ions, and define the box.
  \item Minimize and validate.
\end{enumerate}
\end{frame}

\begin{frame}
\frametitle{Input files}
\begin{itemize}
  \item PDB: coordinates and experimental metadata.
  \item MOL2/SDF: ligands and small molecules.
  \item Topology and parameters are assigned after loading.
\end{itemize}
\end{frame}

\begin{frame}
\frametitle{Essential identifiers}
\begin{align*}
  \text{Residue} &= (\text{chain},\text{number},\text{name}),\\
  \text{Atom} &= (\text{type},\text{element},\text{charge}).
\end{align*}
\begin{itemize}
  \item Consistent identifiers prevent errors in the force field.
\end{itemize}
\end{frame}

\begin{frame}
\frametitle{Reproducibility}
\begin{itemize}
  \item Save preparation scripts.
  \item Document library versions and parameters.
  \item Avoid non-traceable manual steps.
\end{itemize}
\end{frame}

\subsection{Guided modeling}

\begin{frame}
\frametitle{Official modeling workflow}
\begin{itemize}
  \item The OpenMM User Guide (§4.1-4.6) recommends reconnecting hydrogens, adding solvent, and membranes before parameterization.
  \item The examples repo includes \href{https://github.com/openmm/openmm/blob/master/examples/python-examples/simulateAmber.py}{\texttt{simulateAmber.py}} + \href{https://github.com/openmm/openmm/blob/master/examples/python-examples/simulateCharmm.py}{\texttt{simulateCharmm.py}}, and \href{https://github.com/openmm/openmm/blob/master/examples/python-examples/argon-chemical-potential.py}{\texttt{argon-chemical-potential.py}} to validate free energies in simple liquids.
  \item We reuse that infrastructure for alanine dipeptide and the protein–ligand complex, extending scripts with membrane variables or coarse-grained polymers.
\end{itemize}
\end{frame}

\begin{frame}
\frametitle{Additional systems}
\begin{itemize}
  \item \href{https://openmm.github.io/openmm-cookbook/latest/notebooks/tutorials/coarse_grained_polymer.html}{\texttt{coarse\_grained\_polymer.py}} (OpenMM Cookbook) builds bead-spring topologies that show how to define masses and bonds manually.
  \item The same approach links to \href{https://github.com/openmm/openmm/blob/master/examples/python-examples/argon-chemical-potential.py}{\texttt{argon-chemical-potential.py}}, where Lennard-Jones forces are parameterized and insertion free energy is measured.
  \item Biomolecular models benefit from combining these lightweight examples with Amber-style solvation and heavy ligand handling.
\end{itemize}
\end{frame}

\begin{frame}
\frametitle{Membranes and solvents}
\begin{itemize}
  \item Add solvent with `Modeller.addSolvent` and choose OPC/TIP3P (User Guide §4.2).
  \item For membranes, use `Modeller.addMembrane` and then the \href{https://github.com/openmm/openmm/blob/master/examples/python-examples/simulateAmber.py}{\texttt{simulateAmber.py}} script with an anisotropic barostat.
  \item Save the final topology to reproduce the system exactly (OpenMM App §4.6).
\end{itemize}
\end{frame}

\subsection{Structure quality}

\begin{frame}
\frametitle{PDB quality}
\begin{itemize}
  \item Resolution and B-factors indicate uncertainty.
  \item Flexible regions often show gaps or low occupancy.
\end{itemize}
\end{frame}

\begin{frame}
\frametitle{Missing residues}
\begin{itemize}
  \item PDB files may lack entire segments.
  \item Reconstruction requires inferring geometry and stereochemistry.
\end{itemize}
\end{frame}

\begin{frame}
\frametitle{AltLocs and occupancy}
\begin{align*}
  \sum_k \text{occ}_k &\le 1.
\end{align*}
\begin{itemize}
  \item Choose the dominant conformation or average according to the goal.
\end{itemize}
\end{frame}

\begin{frame}
\frametitle{Active site in PDB}
\begin{center}
  \includegraphics[width=0.78\linewidth]{pdb_reaction_site.png}
\end{center}
\end{frame}

\begin{frame}
\frametitle{Superposition and RMSD}
\begin{align*}
  \text{RMSD} &= \sqrt{\frac{1}{N}\sum_{i=1}^N \lVert \mathbf{r}_i-\mathbf{r}_i^{\,\text{ref}} \rVert^2 }.
\end{align*}
\begin{itemize}
  \item Verify consistency with reference structures.
\end{itemize}
\end{frame}

\subsection{Repair and completion}

\begin{frame}
\frametitle{PDBFixer/Modeller}
\begin{itemize}
  \item Fills residues, corrects names, and removes unwanted ligands.
  \item Prepares the system for force field assignment.
\end{itemize}
\end{frame}

\begin{frame}
\frametitle{Residue insertion}
\begin{itemize}
  \item Uses geometric and stereochemical information.
  \item Local minimization relieves clashes.
\end{itemize}
\end{frame}

\begin{frame}
\frametitle{Protonation and pH}
\begin{align*}
  \frac{[A^-]}{[HA]} &= 10^{\mathrm{pH}-\mathrm{p}K_a}.
\end{align*}
\begin{itemize}
  \item Determines charge states of titratable residues.
\end{itemize}
\end{frame}

\begin{frame}
\frametitle{Charge states}
\begin{itemize}
  \item Tune histidines (HID/HIE/HIP), ASP/GLU, LYS/ARG.
  \item Maintain consistency with the active site environment.
\end{itemize}
\end{frame}

\begin{frame}
\frametitle{Structural growth}
\begin{center}
  \includegraphics[width=0.72\linewidth]{pdb_growth.png}
\end{center}
\end{frame}

\subsection{Force fields}

\begin{frame}
\frametitle{Potential decomposition}
\begin{align*}
  U &= U_{\text{bond}} + U_{\text{angle}} + U_{\text{dihedral}} + U_{\text{nonbonded}}.
\end{align*}
\begin{itemize}
  \item Basis for computing forces and energies.
\end{itemize}
\end{frame}

\begin{frame}
\frametitle{Bonds and angles}
\begin{align*}
  U_{\text{bond}} &= \sum_b k_b (r_b-r_b^0)^2,\\
  U_{\text{angle}} &= \sum_a k_a (\theta_a-\theta_a^0)^2.
\end{align*}
\begin{itemize}
  \item Harmonic approximation around the equilibrium.
\end{itemize}
\end{frame}

\begin{frame}
\frametitle{Dihedrals}
\begin{align*}
  U_{\text{dihedro}} &= \sum_d \frac{V_d}{2}\left[1+\cos(n_d\phi_d-\gamma_d)\right].
\end{align*}
\begin{itemize}
  \item Control rotational barriers and conformations.
\end{itemize}
\end{frame}

\begin{frame}
\frametitle{Nonbonded terms}
\begin{align*}
  U_{\text{LJ}} &= 4\epsilon\left[\left(\frac{\sigma}{r}\right)^{12}-\left(\frac{\sigma}{r}\right)^6\right],\\
  U_{\text{C}} &= \frac{1}{4\pi\varepsilon_0}\frac{q_i q_j}{r_{ij}}.
\end{align*}
\begin{itemize}
  \item LJ and Coulomb dominate long-range interactions.
\end{itemize}
\end{frame}

\begin{frame}
\frametitle{Force field selection}
\begin{itemize}
  \item Proteins: AMBER/CHARMM/OPLS.
  \item Ligands: OpenFF or other parameterizers.
  \item Validate compatibility with the rest of the system.
\end{itemize}
\end{frame}

\begin{frame}
\frametitle{Parameter consistency}
\begin{itemize}
  \item Avoid mixing force fields without clear rules.
  \item Check units and energy scales.
\end{itemize}
\end{frame}

\subsection{Ligands and charges}

\begin{frame}
\frametitle{Partial charges}
\begin{align*}
  \sum_i q_i &= Q_{\text{total}}.
\end{align*}
\begin{itemize}
  \item The total must match the protonation state.
\end{itemize}
\end{frame}

\begin{frame}
\frametitle{Electrostatic fitting}
\begin{itemize}
  \item RESP and AM1-BCC derive charges from electrostatic potentials.
  \item The charge distribution affects binding energies.
\end{itemize}
\end{frame}

\begin{frame}
\frametitle{Dihedral scans}
\begin{align*}
  E(\phi) &= E_0 + \sum_k V_k \cos(k\phi-\gamma_k).
\end{align*}
\begin{itemize}
  \item Fit rotational profiles on ligands.
\end{itemize}
\end{frame}

\begin{frame}
\frametitle{Type assignment}
\begin{itemize}
  \item Atomic types determine LJ and bond parameters.
  \item Inconsistencies yield unphysical energies.
\end{itemize}
\end{frame}

\begin{frame}
\frametitle{Ligand validation}
\begin{itemize}
  \item Check geometry, chirality, and net charges.
  \item Compare with experimental or QM references.
\end{itemize}
\end{frame}

\subsection{Solvation}

\begin{frame}
\frametitle{Box choice}
\begin{itemize}
  \item Cubic, orthorhombic, or dodecahedral.
  \item Trade-off between cost and distance to the periodic image.
\end{itemize}
\end{frame}

\begin{frame}
\frametitle{Number of water molecules}
\begin{align*}
  N_{\text{H2O}} &\approx \rho\,V\,\frac{N_A}{M_{\text{H2O}}}.
\end{align*}
\begin{itemize}
  \item Estimate the box size according to the target density.
\end{itemize}
\end{frame}

\begin{frame}
\frametitle{Ionic concentration}
\begin{align*}
  N_{\text{ion}} &= C\,V\,N_A.
\end{align*}
\begin{itemize}
  \item Adjust molarity for physiological conditions.
\end{itemize}
\end{frame}

\begin{frame}
\frametitle{Solvated box}
\begin{center}
  \includegraphics[width=0.7\linewidth]{1i45_wat_box.png}
\end{center}
\end{frame}

\begin{frame}
\frametitle{Water models}
\begin{itemize}
  \item TIP3P, SPC/E, TIP4P: different densities and dynamics.
  \item Choose the model compatible with the force field.
\end{itemize}
\end{frame}

\begin{frame}
\frametitle{Solvent trimming}
\begin{itemize}
  \item Keep a minimum buffer around the solute.
  \item Prevent artificial interactions with the periodic image.
\end{itemize}
\end{frame}

\begin{frame}
\frametitle{Density check}
\begin{itemize}
  \item Verify density after NPT equilibration.
  \item Adjust the size if the density drifts.
\end{itemize}
\end{frame}

\subsection{Ions and electrostatics}

\begin{frame}
\frametitle{Neutralization}
\begin{itemize}
  \item Neutralizing the net charge improves numerical stability.
\end{itemize}
\end{frame}

\begin{frame}
\frametitle{Ionic strength}
\begin{align*}
  I &= \frac{1}{2}\sum_i c_i z_i^2.
\end{align*}
\begin{itemize}
  \item Controls electrostatic screening.
\end{itemize}
\end{frame}

\begin{frame}
\frametitle{Debye length}
\begin{align*}
  \lambda_D &= \sqrt{\frac{\varepsilon \kb T}{2N_A e^2 I}}.
\end{align*}
\begin{itemize}
  \item Sets the range of electrostatic interactions in solution.
\end{itemize}
\end{frame}

\begin{frame}
\frametitle{PME and cutoff}
\begin{itemize}
  \item PME handles long-range Coulomb interactions efficiently.
  \item Choose a cutoff consistent with the box.
\end{itemize}
\end{frame}

\begin{frame}
\frametitle{Charge/solvent balance}
\begin{itemize}
  \item Check that the system is not overloaded with ions.
\end{itemize}
\end{frame}

\subsection{Membranes}

\begin{frame}
\frametitle{Membrane models}
\begin{itemize}
  \item Lipid bilayers require anisotropic boxes.
  \item The orientation must align with the protein.
\end{itemize}
\end{frame}

\begin{frame}
\frametitle{Lipid bilayer}
\begin{center}
  \includegraphics[width=0.72\linewidth]{lipids_bilayer.pdf}
\end{center}
\end{frame}

\begin{frame}
\frametitle{Area per lipid}
\begin{align*}
  A_{\text{lip}} &= \frac{A_{\text{caja}}}{N_{\text{lip}}/2}.
\end{align*}
\begin{itemize}
  \item Key metric for membrane stability.
\end{itemize}
\end{frame}

\begin{frame}
\frametitle{Membrane thickness}
\begin{itemize}
  \item Controls the exposure of hydrophobic domains.
  \item Adjust with minimization and equilibration.
\end{itemize}
\end{frame}

\begin{frame}
\frametitle{Initial restraints}
\begin{itemize}
  \item Restraining the protein prevents collapse during equilibration.
\end{itemize}
\end{frame}

\subsection{Minimization}

\begin{frame}
\frametitle{Objective function}
\begin{align*}
  \min_{\mathbf{r}}\; U(\mathbf{r}).
\end{align*}
\begin{itemize}
  \item Reduces clashes and improves initial stability.
\end{itemize}
\end{frame}

\begin{frame}
\frametitle{Gradient descent}
\begin{align*}
  \mathbf{r}_{k+1} &= \mathbf{r}_k - \alpha_k \nabla U(\mathbf{r}_k).
\end{align*}
\begin{itemize}
  \item Robust method for minimizing high energies.
\end{itemize}
\end{frame}

\begin{frame}
\frametitle{Conjugate gradient}
\begin{itemize}
  \item More efficient near the minimum.
  \item Reduces the number of force evaluations.
\end{itemize}
\end{frame}

\begin{frame}
\frametitle{Harmonic restraints}
\begin{align*}
  U_{\text{rest}} &= \frac{k}{2}\sum_i \lVert \mathbf{r}_i-\mathbf{r}_i^0 \rVert^2.
\end{align*}
\begin{itemize}
  \item Keep the structure while the solvent relaxes.
\end{itemize}
\end{frame}

\begin{frame}
\frametitle{Convergence criteria}
\begin{itemize}
  \item Gradient norms and energy drop.
  \item Stop when forces are small.
\end{itemize}
\end{frame}

\subsection{Validation}

\begin{frame}
\frametitle{Geometric check}
\begin{itemize}
  \item Reasonable bond lengths and angles.
  \item No steric clashes.
\end{itemize}
\end{frame}

\begin{frame}
\frametitle{Energy by component}
\begin{itemize}
  \item Review LJ, Coulomb, and bond contributions.
  \item Identify outliers.
\end{itemize}
\end{frame}

\begin{frame}
\frametitle{Comparison with reference}
\begin{align*}
  \Delta U &= U_{\text{nuevo}}-U_{\text{ref}}.
\end{align*}
\begin{itemize}
  \item Large differences signal parameterization issues.
\end{itemize}
\end{frame}

\begin{frame}
\frametitle{Stability check}
\begin{itemize}
  \item Run a short NVT to catch numerical explosions.
  \item Inspect potential energy and temperature.
\end{itemize}
\end{frame}

\begin{frame}
\frametitle{Stable solvent}
\begin{itemize}
  \item Ensure there are no voids or overlaps.
\end{itemize}
\end{frame}

\begin{frame}
\frametitle{Checklist}
\begin{itemize}
  \item Clean PDB, correct protonation, neutralized system.
  \item Topology file saved.
\end{itemize}
\end{frame}

\subsection{Export}

\begin{frame}
\frametitle{Save the system}
\begin{itemize}
  \item Export PDB and topology with parameters.
  \item Save the minimized state.
\end{itemize}
\end{frame}

\begin{frame}
\frametitle{Output formats}
\begin{itemize}
  \item PDB, Amber prmtop/inpcrd, OpenMM XML.
\end{itemize}
\end{frame}

\begin{frame}
\frametitle{Traceability}
\begin{itemize}
  \item Log scripts and configurations used.
  \item Version the parameters.
\end{itemize}
\end{frame}

\begin{frame}
\frametitle{Seeds and reproducibility}
\begin{itemize}
  \item Save random generator seeds.
  \item Document initial conditions.
\end{itemize}
\end{frame}

\begin{frame}
\frametitle{Folder structure}
\begin{itemize}
  \item Separate inputs, results, and temporary files.
  \item Simplifies auditing and system reuse.
\end{itemize}
\end{frame}

\begin{frame}
\frametitle{Episode summary}
\begin{itemize}
  \item Preparing the system is key for reliable results.
  \item Protonation and solvation decisions impact the dynamics.
  \item Thorough validation avoids costly mistakes.
\end{itemize}
\end{frame}
