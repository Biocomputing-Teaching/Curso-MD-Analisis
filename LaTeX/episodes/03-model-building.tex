\section[Preparacion]{Episodio 3: Preparaci\'on y edici\'on del sistema}

\subsection{Objetivos y flujo}

\begin{frame}
\frametitle{Objetivos del episodio}
\begin{itemize}
  \item Preparar sistemas listos para simular con OpenMM.
  \item Estandarizar topolog\'ias y coordenadas.
  \item Generar entradas reproducibles y verificables.
\end{itemize}
\end{frame}

\begin{frame}
\frametitle{Flujo de preparaci\'on}
\begin{enumerate}
  \item Lectura de estructura (PDB/MOL2/SDF).
  \item Reparaci\'on y completado de residuos.
  \item Protonaci\'on y asignaci\'on de cargas.
  \item Solvataci\'on, iones y caja.
  \item Minimizado y validaci\'on.
\end{enumerate}
\end{frame}

\begin{frame}
\frametitle{Archivos de entrada}
\begin{itemize}
  \item PDB: coordenadas y metadatos experimentales.
  \item MOL2/SDF: ligandos y peque\~nas mol\'eculas.
  \item Topolog\'ia y par\'ametros se asignan despu\'es de la lectura.
\end{itemize}
\end{frame}

\begin{frame}
\frametitle{Identificadores esenciales}
\begin{align*}
  \text{Residuo} &= (\text{cadena},\text{n\'umero},\text{nombre}),\\
  \text{\'Atomo} &= (\text{tipo},\text{elemento},\text{carga}).
\end{align*}
\begin{itemize}
  \item Los identificadores consistentes evitan errores en el campo de fuerza.
\end{itemize}
\end{frame}

\begin{frame}
\frametitle{Reproducibilidad}
\begin{itemize}
  \item Guardar scripts de preparaci\'on.
  \item Documentar versiones de librer\'ias y par\'ametros.
  \item Evitar pasos manuales no trazables.
\end{itemize}
\end{frame}

\subsection{Modelado guiado}

\begin{frame}
\frametitle{Modelado conforme a la gu\'ia oficial}
\begin{itemize}
  \item OpenMM User Guide (§4.1-4.6) sugiere reconectar hidrógenos, solvatación y adición de membranas antes de parametrizar.
  \item El repositorio de ejemplos incluye \href{https://github.com/openmm/openmm/blob/master/examples/python-examples/simulateAmber.py}{\texttt{simulateAmber.py}} + \href{https://github.com/openmm/openmm/blob/master/examples/python-examples/simulateCharmm.py}{\texttt{simulateCharmm.py}}, y \href{https://github.com/openmm/openmm/blob/master/examples/python-examples/argon-chemical-potential.py}{\texttt{argon-chemical-potential.py}} para verificar energías libres en líquidos simples.
  \item Podemos reutilizar la infraestructura para el dipéptido de alanina y el complejo proteína-ligando, pero extendiendo los scripts con variables de membrana o polímeros coarse-grained.
\end{itemize}
\end{frame}

\begin{frame}
\frametitle{Sistemas adicionales}
\begin{itemize}
  \item \href{https://openmm.github.io/openmm-cookbook/latest/notebooks/tutorials/coarse_grained_polymer.html}{\texttt{coarse\_grained\_polymer.py}} (OpenMM Cookbook) crea topologías bead-spring que muestran cómo definir masas y enlaces manualmente.
  \item Ese mismo enfoque se conecta con el script \href{https://github.com/openmm/openmm/blob/master/examples/python-examples/argon-chemical-potential.py}{\texttt{argon-chemical-potential.py}}, donde se parametrizan fuerzas de Lennard-Jones y se mide energía libre de inserción.
  \item Los modelos biomoleculares se benefician al combinar estos ejemplos ligeros con la solvatación y el tratamiento de ligandos pesados en Amber.
\end{itemize}
\end{frame}

\begin{frame}
\frametitle{Membranas y solventes}
\begin{itemize}
  \item Añadir solvente con `Modeller.addSolvent` y escoger OPC/TIP3P (User Guide §4.2).
  \item Si hay membranas, usar `Modeller.addMembrane` y luego \href{https://github.com/openmm/openmm/blob/master/examples/python-examples/simulateAmber.py}{\texttt{simulateAmber.py}} con barostato anisotrópico.
  \item Guardar la topología final para reproducir exactamente el sistema (OpenMM App §4.6).
\end{itemize}
\end{frame}

\subsection{Calidad de la estructura}

\begin{frame}
\frametitle{Calidad del PDB}
\begin{itemize}
  \item Resoluci\'on y factores B informan sobre la incertidumbre.
  \item Regiones flexibles suelen presentar huecos o baja ocupaci\'on.
\end{itemize}
\end{frame}

\begin{frame}
\frametitle{Residuos faltantes}
\begin{itemize}
  \item Los PDB pueden carecer de segmentos completos.
  \item La reconstrucci\'on requiere inferir geometr\'ia y estereoq\'u\'imica.
\end{itemize}
\end{frame}

\begin{frame}
\frametitle{AltLocs y ocupaci\'on}
\begin{align*}
  \sum_k \text{occ}_k &\le 1.
\end{align*}
\begin{itemize}
  \item Elegir la conformaci\'on dominante o promediar seg\'un objetivo.
\end{itemize}
\end{frame}

\begin{frame}
\frametitle{Sitio activo en PDB}
\begin{center}
  \includegraphics[width=0.78\linewidth]{episodes/pdb_reaction_site.png}
\end{center}
\end{frame}

\begin{frame}
\frametitle{Superposici\'on y RMSD}
\begin{align*}
  \text{RMSD} &= \sqrt{\frac{1}{N}\sum_{i=1}^N \lVert \mathbf{r}_i-\mathbf{r}_i^{\,\text{ref}} \rVert^2 }.
\end{align*}
\begin{itemize}
  \item Verifica coherencia con estructuras de referencia.
\end{itemize}
\end{frame}

\subsection{Reparaci\'on y completado}

\begin{frame}
\frametitle{PDBFixer/Modeller}
\begin{itemize}
  \item Rellena residuos, corrige nombres y quita ligandos no deseados.
  \item Prepara el sistema para asignar campo de fuerza.
\end{itemize}
\end{frame}

\begin{frame}
\frametitle{Inserci\'on de residuos}
\begin{itemize}
  \item Se usa informaci\'on geom\'etrica y estereoq\'u\'imica.
  \item Minimizado local para aliviar choques.
\end{itemize}
\end{frame}

\begin{frame}
\frametitle{Protonaci\'on y pH}
\begin{align*}
  \frac{[A^-]}{[HA]} &= 10^{\mathrm{pH}-\mathrm{p}K_a}.
\end{align*}
\begin{itemize}
  \item Determina estados de carga de residuos titulables.
\end{itemize}
\end{frame}

\begin{frame}
\frametitle{Estados de carga}
\begin{itemize}
  \item Ajustar histidinas (HID/HIE/HIP), ASP/GLU, LYS/ARG.
  \item Coherencia con el entorno del sitio activo.
\end{itemize}
\end{frame}

\begin{frame}
\frametitle{Crecimiento estructural}
\begin{center}
  \includegraphics[width=0.72\linewidth]{episodes/pdb_growth.png}
\end{center}
\end{frame}

\subsection{Campos de fuerza}

\begin{frame}
\frametitle{Descomposici\'on del potencial}
\begin{align*}
  U &= U_{\text{enlace}} + U_{\text{\'angulo}} + U_{\text{dihedro}} + U_{\text{no enlazado}}.
\end{align*}
\begin{itemize}
  \item Base del c\'alculo de fuerzas y energ\'ias.
\end{itemize}
\end{frame}

\begin{frame}
\frametitle{Enlaces y \`angulos}
\begin{align*}
  U_{\text{enlace}} &= \sum_b k_b (r_b-r_b^0)^2,\\
  U_{\text{\'angulo}} &= \sum_a k_a (\theta_a-\theta_a^0)^2.
\end{align*}
\begin{itemize}
  \item Aproximaci\'on arm\'onica alrededor del equilibrio.
\end{itemize}
\end{frame}

\begin{frame}
\frametitle{Dihedros}
\begin{align*}
  U_{\text{dihedro}} &= \sum_d \frac{V_d}{2}\left[1+\cos(n_d\phi_d-\gamma_d)\right].
\end{align*}
\begin{itemize}
  \item Controlan barreras rotacionales y conformaciones.
\end{itemize}
\end{frame}

\begin{frame}
\frametitle{No enlazados}
\begin{align*}
  U_{\text{LJ}} &= 4\epsilon\left[\left(\frac{\sigma}{r}\right)^{12}-\left(\frac{\sigma}{r}\right)^6\right],\\
  U_{\text{C}} &= \frac{1}{4\pi\varepsilon_0}\frac{q_i q_j}{r_{ij}}.
\end{align*}
\begin{itemize}
  \item LJ y Coulomb dominan interacciones a distancia.
\end{itemize}
\end{frame}

\begin{frame}
\frametitle{Selecci\'on del campo de fuerza}
\begin{itemize}
  \item Prote\'inas: AMBER/CHARMM/OPLS.
  \item Ligandos: OpenFF u otros parametrizadores.
  \item Validar compatibilidad con el sistema.
\end{itemize}
\end{frame}

\begin{frame}
\frametitle{Consistencia de par\'ametros}
\begin{itemize}
  \item Evitar mezclar campos de fuerza sin reglas claras.
  \item Revisar unidades y escalas de energ\'ia.
\end{itemize}
\end{frame}

\subsection{Ligandos y cargas}

\begin{frame}
\frametitle{Cargas parciales}
\begin{align*}
  \sum_i q_i &= Q_{\text{total}}.
\end{align*}
\begin{itemize}
  \item El total debe coincidir con el estado de protonaci\'on.
\end{itemize}
\end{frame}

\begin{frame}
\frametitle{Ajuste electrost\'atico}
\begin{itemize}
  \item RESP y AM1-BCC derivan cargas de potenciales electrost\'aticos.
  \item La distribuci\'on de cargas afecta energ\'ias de uni\'on.
\end{itemize}
\end{frame}

\begin{frame}
\frametitle{Escaneo de dihedros}
\begin{align*}
  E(\phi) &= E_0 + \sum_k V_k \cos(k\phi-\gamma_k).
\end{align*}
\begin{itemize}
  \item Ajusta perfiles rotacionales en ligandos.
\end{itemize}
\end{frame}

\begin{frame}
\frametitle{Asignaci\'on de tipos}
\begin{itemize}
  \item Tipos at\'omicos determinan par\'ametros LJ y enlaces.
  \item Inconsistencias producen energ\'ias no f\'isicas.
\end{itemize}
\end{frame}

\begin{frame}
\frametitle{Validaci\'on del ligando}
\begin{itemize}
  \item Revisar geometr\'ia, quiralidad y cargas netas.
  \item Comparar con referencias experimentales o QM.
\end{itemize}
\end{frame}

\subsection{Solvataci\'on}

\begin{frame}
\frametitle{Elecci\'on de caja}
\begin{itemize}
  \item C\'ubica, ortorr\'ombica o dodeca\'edrica.
  \item Compromiso entre coste y distancia a la imagen.
\end{itemize}
\end{frame}

\begin{frame}
\frametitle{N\'umero de mol\'eculas de agua}
\begin{align*}
  N_{\text{H2O}} &\approx \rho\,V\,\frac{N_A}{M_{\text{H2O}}}.
\end{align*}
\begin{itemize}
  \item Estima el tama\~no de caja seg\'un densidad objetivo.
\end{itemize}
\end{frame}

\begin{frame}
\frametitle{Concentraci\'on i\'onica}
\begin{align*}
  N_{\text{ion}} &= C\,V\,N_A.
\end{align*}
\begin{itemize}
  \item Ajustar molaridad para condiciones fisiol\'ogicas.
\end{itemize}
\end{frame}

\begin{frame}
\frametitle{Caja solvatada}
\begin{center}
  \includegraphics[width=0.7\linewidth]{episodes/1i45_wat_box.png}
\end{center}
\end{frame}

\begin{frame}
\frametitle{Modelos de agua}
\begin{itemize}
  \item TIP3P, SPC/E, TIP4P: distintas densidades y din\'amicas.
  \item Elegir el modelo compatible con el campo de fuerza.
\end{itemize}
\end{frame}

\begin{frame}
\frametitle{Recorte de solvente}
\begin{itemize}
  \item Mantener un colch\'on m\'inimo alrededor del soluto.
  \item Evita interacciones artificiales con la imagen peri\'odica.
\end{itemize}
\end{frame}

\begin{frame}
\frametitle{Chequeo de densidad}
\begin{itemize}
  \item Verificar densidad tras equilibrado NPT.
  \item Ajustar tama\~no si la densidad se desv\'ia.
\end{itemize}
\end{frame}

\subsection{Iones y electrost\'atica}

\begin{frame}
\frametitle{Neutralizaci\'on}
\begin{itemize}
  \item Neutralizar la carga neta mejora estabilidad num\'erica.
\end{itemize}
\end{frame}

\begin{frame}
\frametitle{Fuerza i\'onica}
\begin{align*}
  I &= \frac{1}{2}\sum_i c_i z_i^2.
\end{align*}
\begin{itemize}
  \item Controla apantallamiento electrost\'atico.
\end{itemize}
\end{frame}

\begin{frame}
\frametitle{Longitud de Debye}
\begin{align*}
  \lambda_D &= \sqrt{\frac{\varepsilon \kb T}{2N_A e^2 I}}.
\end{align*}
\begin{itemize}
  \item Define el rango de interacciones electrost\'aticas en soluci\'on.
\end{itemize}
\end{frame}

\begin{frame}
\frametitle{PME y corte}
\begin{itemize}
  \item PME trata Coulomb de largo alcance de forma eficiente.
  \item Elegir corte consistente con la caja.
\end{itemize}
\end{frame}

\begin{frame}
\frametitle{Balance carga/solvente}
\begin{itemize}
  \item Comprobar que el sistema no est\'e sobrecargado de iones.
\end{itemize}
\end{frame}

\subsection{Membranas}

\begin{frame}
\frametitle{Modelos de membrana}
\begin{itemize}
  \item Bicapas lip\'idicas requieren cajas anisotr\'opicas.
  \item La orientaci\'on debe ser coherente con la prote\'ina.
\end{itemize}
\end{frame}

\begin{frame}
\frametitle{Bicapa lip\'idica}
\begin{center}
  \includegraphics[width=0.72\linewidth]{episodes/lipids_bilayer.pdf}
\end{center}
\end{frame}

\begin{frame}
\frametitle{Area por l\'ipido}
\begin{align*}
  A_{\text{lip}} &= \frac{A_{\text{caja}}}{N_{\text{lip}}/2}.
\end{align*}
\begin{itemize}
  \item M\'etrica clave para estabilidad de la membrana.
\end{itemize}
\end{frame}

\begin{frame}
\frametitle{Espesor de membrana}
\begin{itemize}
  \item Controla la exposici\'on de dominios hidrof\'obicos.
  \item Ajuste con minimizado y equilibrado.
\end{itemize}
\end{frame}

\begin{frame}
\frametitle{Restricciones iniciales}
\begin{itemize}
  \item Restringir la prote\'ina evita colapsos durante equilibrado.
\end{itemize}
\end{frame}

\subsection{Minimizado}

\begin{frame}
\frametitle{Funci\'on objetivo}
\begin{align*}
  \min_{\mathbf{r}}\; U(\mathbf{r}).
\end{align*}
\begin{itemize}
  \item Reduce choques y mejora la estabilidad inicial.
\end{itemize}
\end{frame}

\begin{frame}
\frametitle{Descenso del gradiente}
\begin{align*}
  \mathbf{r}_{k+1} &= \mathbf{r}_k - \alpha_k \nabla U(\mathbf{r}_k).
\end{align*}
\begin{itemize}
  \item M\'etodo robusto para minimizar energ\'ias altas.
\end{itemize}
\end{frame}

\begin{frame}
\frametitle{Gradiente conjugado}
\begin{itemize}
  \item M\'as eficiente cerca del m\'inimo.
  \item Reduce el n\'umero de evaluaciones de fuerza.
\end{itemize}
\end{frame}

\begin{frame}
\frametitle{Restricciones arm\'onicas}
\begin{align*}
  U_{\text{rest}} &= \frac{k}{2}\sum_i \lVert \mathbf{r}_i-\mathbf{r}_i^0 \rVert^2.
\end{align*}
\begin{itemize}
  \item Mantienen la estructura mientras se relaja el solvente.
\end{itemize}
\end{frame}

\begin{frame}
\frametitle{Criterios de convergencia}
\begin{itemize}
  \item Normas del gradiente y disminuci\'on de energ\'ia.
  \item Detener cuando las fuerzas son peque\~nas.
\end{itemize}
\end{frame}

\subsection{Validaci\'on}

\begin{frame}
\frametitle{Chequeo geom\'etrico}
\begin{itemize}
  \item Longitudes y \`angulos razonables.
  \item Ausencia de choques est\'ericos.
\end{itemize}
\end{frame}

\begin{frame}
\frametitle{Energia por componente}
\begin{itemize}
  \item Revisar contribuciones LJ, Coulomb y enlaces.
  \item Identificar valores extremos.
\end{itemize}
\end{frame}

\begin{frame}
\frametitle{Comparaci\'on con referencia}
\begin{align*}
  \Delta U &= U_{\text{nuevo}}-U_{\text{ref}}.
\end{align*}
\begin{itemize}
  \item Diferencias grandes indican problemas de parametrizaci\'on.
\end{itemize}
\end{frame}

\begin{frame}
\frametitle{Chequeo de estabilidad}
\begin{itemize}
  \item Ejecutar un NVT corto para detectar explosiones num\'ericas.
  \item Inspeccionar energ\'ia potencial y temperatura.
\end{itemize}
\end{frame}

\begin{frame}
\frametitle{Solvente estable}
\begin{itemize}
  \item Verificar que no hay vac\'ios ni solapamientos.
\end{itemize}
\end{frame}

\begin{frame}
\frametitle{Lista de control}
\begin{itemize}
  \item PDB limpio, protonaci\'on correcta, sistema neutralizado.
  \item Archivo de topolog\'ia guardado.
\end{itemize}
\end{frame}

\subsection{Exportaci\'on}

\begin{frame}
\frametitle{Guardar el sistema}
\begin{itemize}
  \item Exportar PDB y topolog\'ia con par\'ametros.
  \item Guardar estado de minimizado.
\end{itemize}
\end{frame}

\begin{frame}
\frametitle{Formatos de salida}
\begin{itemize}
  \item PDB, Amber prmtop/inpcrd, OpenMM XML.
\end{itemize}
\end{frame}

\begin{frame}
\frametitle{Trazabilidad}
\begin{itemize}
  \item Registrar scripts y configuraciones usadas.
  \item Versionar los par\'ametros.
\end{itemize}
\end{frame}

\begin{frame}
\frametitle{Semillas y reproducibilidad}
\begin{itemize}
  \item Guardar semillas de generadores aleatorios.
  \item Documentar condiciones iniciales.
\end{itemize}
\end{frame}

\begin{frame}
\frametitle{Estructura de carpetas}
\begin{itemize}
  \item Separar entradas, resultados y archivos temporales.
  \item Facilita auditor\\'ia y reutilizaci\\'on del sistema.
\end{itemize}
\end{frame}

\begin{frame}
\frametitle{Resumen del episodio}
\begin{itemize}
  \item Preparar el sistema es clave para resultados fiables.
  \item Las decisiones en protonaci\'on y solvataci\'on impactan la din\'amica.
  \item Una buena validaci\'on evita errores costosos.
\end{itemize}
\end{frame}
