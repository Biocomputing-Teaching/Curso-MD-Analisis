\section[Simulaciones]{Episodio 2: Ejecuci\'on de simulaciones}

\subsection{Par\'ametros clave (OpenMM User Guide §3.7)}

\begin{frame}
\frametitle{Compromisos velocidad vs. exactitud}
\begin{itemize}
  \item Velocidad: pasos grandes y menos reporting.
  \item Exactitud num\'erica: integradores simp\'ecticos, restricciones, timestep pequeño.
  \item OpenMM ofrece sliders para \texttt{dt}, \texttt{constraints}, \texttt{hydrogenMass} y termostatos (§3.7.6-3.7.8).
\end{itemize}
\end{frame}

\begin{frame}
\frametitle{Paso y restricciones}
\begin{align*}
  \Delta t &\approx \frac{1}{10\,\omega_{\text{max}}},\quad \omega_{\text{max}}\sim\sqrt{\frac{k}{\mu}}.
\end{align*}
\begin{itemize}
  \item Con \texttt{constraints=HBonds}, se duplica el timestep.
  \item \texttt{constraints=AllBonds/HAngles} permiten más salto pero reducen flexibilidad.
\end{itemize}
\end{frame}

\begin{frame}
\frametitle{Hidrógenos pesados}
\begin{align*}
  m_H^{\text{new}} &= \alpha\,m_H,\quad m_{heavy}^{\text{new}} = m_{heavy} - (\alpha-1)m_H.
\end{align*}
\begin{itemize}
  \item \texttt{hydrogenMass=1.5*amu} ralentiza oscilaciones sin perder masa total.
  \item Utilizado en scripts de integración, p.ej. el ejemplo \href{https://github.com/openmm/openmm/blob/master/examples/python-examples/argon-chemical-potential.py}{\texttt{argon-chemical-potential.py}} adapta este truco.
\end{itemize}
\end{frame}

\begin{frame}
\frametitle{Acoplamientos térmicos y de presión}
\begin{itemize}
  \item \texttt{LangevinIntegrator} simula un baño de calor con fricción $\gamma$.
  \item Nosé-Hoover (cadena) controla distribución canónica manteniendo el momento.
  \item Barostato de Monte Carlo o Parrinello-Rahman aparece en los scripts de \href{https://github.com/openmm/openmm/blob/master/examples/python-examples/simulateAmber.py}{\texttt{simulateAmber.py}}.
\end{itemize}
\end{frame}

\begin{frame}
\frametitle{Reportes y checkpoints}
\begin{itemize}
  \item `StateDataReporter` controla energía, temperatura, volumen y RMSD.
  \item `DCDReporter` / `NetCDFReporter` guardan trayectorias (OpenMM App §3.13-3.14).
  \item Usa `CheckpointReporter` para reiniciar simulaciones largas (OpenMM §3.15).
\end{itemize}
\end{frame}

\subsection{Discretizaci\'on temporal}

\begin{frame}
\frametitle{Ecuaciones continuas y discretas}
\begin{align*}
  \dot{\mathbf{r}}(t) &= \mathbf{v}(t),\quad \dot{\mathbf{v}}(t) = \frac{\mathbf{F}(t)}{m}.
\end{align*}
\begin{itemize}
  \item La simulaci\'on sustituye derivadas por incrementos finitos.
\end{itemize}
\end{frame}

\begin{frame}
\frametitle{Paso de tiempo}
\begin{align*}
  \mathbf{r}(t+\Delta t) &= \mathbf{r}(t) + \mathbf{v}(t)\Delta t + \mathcal{O}(\Delta t^2).
\end{align*}
\begin{itemize}
  \item $\Delta t$ debe resolver las vibraciones m\'as r\'apidas.
\end{itemize}
\end{frame}

\begin{frame}
\frametitle{Error local y global}
\begin{align*}
  \text{error local} &= \mathcal{O}(\Delta t^{p+1}),\\
  \text{error global} &= \mathcal{O}(\Delta t^{p}).
\end{align*}
\begin{itemize}
  \item Los integradores de orden $p$ controlan la acumulaci\'on de error.
\end{itemize}
\end{frame}

\begin{frame}
\frametitle{Escala de estabilidad}
\begin{align*}
  \Delta t &\lesssim \frac{1}{10\,\omega_{\text{max}}}.
\end{align*}
\begin{itemize}
  \item $\omega_{\text{max}}$ proviene de los modos vibracionales m\'as r\'apidos.
\end{itemize}
\end{frame}

\begin{frame}
\frametitle{Criterio pr\'actico}
\begin{itemize}
  \item Sin restricciones: $\Delta t \approx 1$ fs.
  \item Con SHAKE/RATTLE: $\Delta t \approx 2$ fs.
\end{itemize}
\end{frame}

\subsection{Integradores cl\'asicos}

\begin{frame}
\frametitle{Euler expl\'icito}
\begin{align*}
  \mathbf{v}_{t+\Delta t} &= \mathbf{v}_t + \frac{\mathbf{F}_t}{m}\Delta t,\\
  \mathbf{r}_{t+\Delta t} &= \mathbf{r}_t + \mathbf{v}_t\Delta t.
\end{align*}
\begin{itemize}
  \item Simple pero inestable para din\'amica molecular.
\end{itemize}
\end{frame}

\begin{frame}
\frametitle{Verlet cl\'asico}
\begin{align*}
  \mathbf{r}_{t+\Delta t} &= 2\mathbf{r}_t - \mathbf{r}_{t-\Delta t} + \frac{\mathbf{F}_t}{m}\Delta t^2.
\end{align*}
\begin{itemize}
  \item Sim\'etrico y reversible en el tiempo.
\end{itemize}
\end{frame}

\begin{frame}
\frametitle{Velocity Verlet}
\begin{align*}
  \mathbf{r}_{t+\Delta t} &= \mathbf{r}_t + \mathbf{v}_t\Delta t + \frac{\mathbf{F}_t}{2m}\Delta t^2,\\
  \mathbf{v}_{t+\Delta t} &= \mathbf{v}_t + \frac{\mathbf{F}_t+\mathbf{F}_{t+\Delta t}}{2m}\Delta t.
\end{align*}
\begin{itemize}
  \item Combina estabilidad y acceso directo a velocidades.
\end{itemize}
\end{frame}

\begin{frame}
\frametitle{Simetr\'ia y estabilidad}
\begin{itemize}
  \item Integradores simp\'ecticos conservan volumen en el espacio de fases.
  \item Evitan deriva energ\'etica en NVE.
\end{itemize}
\end{frame}

\begin{frame}
\frametitle{Elecci\'on pr\'actica}
\begin{itemize}
  \item OpenMM usa integradores tipo Verlet o Langevin.
  \item La elecci\'on depende del ensemble deseado.
\end{itemize}
\end{frame}

\subsection{Esquema leapfrog}

\begin{frame}
\frametitle{Leapfrog}
\begin{align*}
  \mathbf{v}_{t+\Delta t/2} &= \mathbf{v}_{t-\Delta t/2} + \frac{\mathbf{F}_t}{m}\Delta t,\\
  \mathbf{r}_{t+\Delta t} &= \mathbf{r}_t + \mathbf{v}_{t+\Delta t/2}\Delta t.
\end{align*}
\begin{itemize}
  \item Velocidades y posiciones ``se alternan'' en el tiempo.
\end{itemize}
\end{frame}

\begin{frame}
\frametitle{Interpretaci\'on gr\'afica}
\begin{center}
  \includegraphics[width=0.72\linewidth]{episodes/leapfrog_method_argument.png}
\end{center}
\vspace{-0.1cm}
{\tiny Fuente: Wikimedia Commons (CC BY-SA 4.0). \cita{img-leapfrog}}
\end{frame}

\begin{frame}
\frametitle{Ventajas}
\begin{itemize}
  \item Sim\'etrico, reversible y de bajo coste.
  \item Conserva mejor la energ\'ia que Euler.
\end{itemize}
\end{frame}

\begin{frame}
\frametitle{Relaci\'on con Verlet}
\begin{itemize}
  \item Leapfrog y velocity Verlet son algebraicamente equivalentes.
  \item Diferencias en el almacenamiento de velocidades.
\end{itemize}
\end{frame}

\begin{frame}
\frametitle{Elecci\'on de integrador}
\begin{itemize}
  \item NVE: Verlet/Leapfrog.
  \item NVT: Langevin o Nos\'e-Hoover.
\end{itemize}
\end{frame}

\subsection{Conservaci\'on y NVE}

\begin{frame}
\frametitle{Energ\'ia total}
\begin{align*}
  E(t) &= K(t) + U(t).
\end{align*}
\begin{itemize}
  \item En NVE ideal, $E(t)$ es constante.
\end{itemize}
\end{frame}

\begin{frame}
\frametitle{Deriva energ\'etica}
\begin{itemize}
  \item Errores num\'ericos producen deriva secular.
  \item Se controla reduciendo $\Delta t$.
\end{itemize}
\end{frame}

\begin{frame}
\frametitle{Temperatura instant\'anea}
\begin{align*}
  T &= \frac{2\langle K \rangle}{3N\kb}.
\end{align*}
\begin{itemize}
  \item En NVE, $T$ fluct\'ua alrededor de un valor medio.
\end{itemize}
\end{frame}

\begin{frame}
\frametitle{Comprobaciones r\'apidas}
\begin{itemize}
  \item Estabilidad de $E(t)$ y distribuci\'on de velocidades.
  \item Comparar $\langle K \rangle$ con el valor te\'orico.
\end{itemize}
\end{frame}

\begin{frame}
\frametitle{Tiempo de equilibraci\'on}
\begin{itemize}
  \item Un pre-equilibrado estabiliza la energ\'ia antes de producir datos.
\end{itemize}
\end{frame}

\subsection{Termostatos}

\begin{frame}
\frametitle{Objetivo del termostato}
\begin{itemize}
  \item Imponer una distribuci\'on can\'onica a temperatura $T$.
  \item Extraer o inyectar energ\'ia de forma controlada.
\end{itemize}
\end{frame}

\begin{frame}
\frametitle{Din\'amica de Langevin}
\begin{align*}
  m\ddot{\mathbf{r}} &= -\nabla U - \gamma m \dot{\mathbf{r}} + \mathbf{R}(t),\\
  \langle \mathbf{R}(t) \mathbf{R}(t') \rangle &= 2\gamma m\kb T\,\delta(t-t').
\end{align*}
\begin{itemize}
  \item Termostato estoc\'astico con fricci\'on y ruido.
\end{itemize}
\end{frame}

\begin{frame}
\frametitle{Andersen}
\begin{itemize}
  \item Reasigna velocidades aleatoriamente con frecuencia $\nu$.
  \item Bueno para muestreo, menos realista din\'amicamente.
\end{itemize}
\end{frame}

\begin{frame}
\frametitle{Nos\'e-Hoover}
\begin{align*}
  \dot{\mathbf{p}}_i &= \mathbf{F}_i - \xi\mathbf{p}_i,\quad \dot{\xi} = \frac{1}{Q}\left(\sum_i \frac{\mathbf{p}_i^2}{m_i} - 3N\kb T\right).
\end{align*}
\begin{itemize}
  \item Termostato determinista con variable extendida.
\end{itemize}
\end{frame}

\begin{frame}
\frametitle{Elecci\'on del termostato}
\begin{itemize}
  \item Langevin: robusto y estable.
  \item Nos\'e-Hoover: din\'amica m\'as realista si est\'a bien parametrizado.
\end{itemize}
\end{frame}

\subsection{Barostatos}

\begin{frame}
\frametitle{Presi\'on microsc\'opica}
\begin{align*}
  P &= \frac{Nk_B T}{V} + \frac{1}{3V}\sum_{i<j}\mathbf{r}_{ij}\cdot\mathbf{F}_{ij}.
\end{align*}
\begin{itemize}
  \item La presi\'on depende de energ\'ia cin\'etica y fuerzas internas.
\end{itemize}
\end{frame}

\begin{frame}
\frametitle{Barostato de Berendsen}
\begin{align*}
  \frac{dV}{dt} &= \frac{1}{\tau_P}(P_0-P)V.
\end{align*}
\begin{itemize}
  \item R\'apido para equilibrar, no reproduce fluctuaciones exactas.
\end{itemize}
\end{frame}

\begin{frame}
\frametitle{Parrinello-Rahman}
\begin{itemize}
  \item Escala la celda de simulaci\'on con variables din\'amicas.
  \item Permite cambios anisotr\'opicos del volumen.
\end{itemize}
\end{frame}

\begin{frame}
\frametitle{NPT}
\begin{itemize}
  \item Ensemble realista para condiciones experimentales.
  \item Combina termostato y barostato.
\end{itemize}
\end{frame}

\begin{frame}
\frametitle{Elecci\'on pr\'actica}
\begin{itemize}
  \item Equilibraci\'on: Berendsen + Langevin.
  \item Producci\'on: barostato m\'as riguroso (Monte Carlo barostat).
\end{itemize}
\end{frame}

\subsection{Restricciones}

\begin{frame}
\frametitle{Motivaci\'on}
\begin{itemize}
  \item Eliminar vibraciones r\'apidas permite aumentar $\Delta t$.
\end{itemize}
\end{frame}

\begin{frame}
\frametitle{Restricciones hol\'onomas}
\begin{align*}
  g_k(\mathbf{r}) &= 0,\quad k=1,\dots,M.
\end{align*}
\begin{itemize}
  \item Fijan distancias o \`angulos internos.
\end{itemize}
\end{frame}

\begin{frame}
\frametitle{SHAKE/RATTLE}
\begin{itemize}
  \item SHAKE corrige posiciones; RATTLE corrige posiciones y velocidades.
\end{itemize}
\end{frame}

\begin{frame}
\frametitle{Impacto num\'erico}
\begin{itemize}
  \item Aumenta estabilidad y reduce coste de simulaci\'on.
  \item Puede alterar modos de alta frecuencia.
\end{itemize}
\end{frame}

\begin{frame}
\frametitle{Recomendaci\'on}
\begin{itemize}
  \item Usar restricciones en enlaces con H para biomol\'eculas.
\end{itemize}
\end{frame}

\subsection{Condiciones peri\'odicas}

\begin{frame}
\frametitle{Motivaci\'on de PBC}
\begin{itemize}
  \item Evita efectos de superficie en sistemas finitos.
  \item Emula un sistema infinito por replicaci\'on de la celda.
\end{itemize}
\end{frame}

\begin{frame}
\frametitle{Celda peri\'odica}
\begin{align*}
  \mathbf{r} &\to \mathbf{r} + n_x\mathbf{a} + n_y\mathbf{b} + n_z\mathbf{c}.
\end{align*}
\begin{itemize}
  \item $\mathbf{a},\mathbf{b},\mathbf{c}$ definen la caja de simulaci\'on.
\end{itemize}
\end{frame}

\begin{frame}
\frametitle{Ejemplo 2D}
\begin{center}
  \includegraphics[width=0.72\linewidth]{episodes/pbc_2d.png}
\end{center}
\vspace{-0.1cm}
{\tiny Fuente: Wikimedia Commons (CC BY-SA 4.0). \cita{img-pbc-2d}}
\end{frame}

\begin{frame}
\frametitle{Implicaciones en din\'amica}
\begin{itemize}
  \item Las part\'iculas que salen reingresan por la cara opuesta.
  \item Se preserva la densidad en el volumen simulado.
\end{itemize}
\end{frame}

\begin{frame}
\frametitle{Cajas no ortogonales}
\begin{itemize}
  \item Celdas tricl\'inicas para cristales o membranas inclinadas.
\end{itemize}
\end{frame}

\subsection{Minimum image}

\begin{frame}
\frametitle{Convenci\'on de imagen m\'inima}
\begin{align*}
  r_{ij} &= \min_{\mathbf{n}} \lVert \mathbf{r}_i-\mathbf{r}_j+\mathbf{n} \rVert.
\end{align*}
\begin{itemize}
  \item Usa la imagen peri\'odica m\'as cercana para calcular interacciones.
\end{itemize}
\end{frame}

\begin{frame}
\frametitle{Ejemplo visual}
\begin{center}
  \includegraphics[width=0.7\linewidth]{episodes/minimum_image_convention.png}
\end{center}
\vspace{-0.1cm}
{\tiny Fuente: Wikimedia Commons (CC BY-SA 4.0). \cita{img-min-image}}
\end{frame}

\begin{frame}
\frametitle{Radio de corte}
\begin{itemize}
  \item La imagen m\'inima requiere $r_c < L/2$ para evitar dobles conteos.
\end{itemize}
\end{frame}

\begin{frame}
\frametitle{Impacto en energ\'ia}
\begin{itemize}
  \item Errores si el corte es grande o la densidad baja.
\end{itemize}
\end{frame}

\begin{frame}
\frametitle{Buenas pr\'acticas}
\begin{itemize}
  \item Ajustar $r_c$ y tama\~no de caja seg\'un la densidad del sistema.
\end{itemize}
\end{frame}

\subsection{Cutoffs y suavizado}

\begin{frame}
\frametitle{Interacciones de corto alcance}
\begin{itemize}
  \item LJ se trunca a $r_c$ con o sin funci\'on de suavizado.
\end{itemize}
\end{frame}

\begin{frame}
\frametitle{Funci\'on switch}
\begin{align*}
  u_{\text{switch}}(r) &= s(r)\,u(r),\quad s(r)\in[0,1].
\end{align*}
\begin{itemize}
  \item Evita discontinuidades en fuerzas.
\end{itemize}
\end{frame}

\begin{frame}
\frametitle{Correcci\'on de cola}
\begin{align*}
  U_{\text{tail}} &\approx 2\pi\rho \int_{r_c}^{\infty} r^2 u(r)\,dr.
\end{align*}
\begin{itemize}
  \item Corrige energ\'ia y presi\'on al truncar LJ.
\end{itemize}
\end{frame}

\begin{frame}
\frametitle{Escogiendo $r_c$}
\begin{itemize}
  \item Mayor $r_c$ mejora precisi\'on, pero aumenta coste.
\end{itemize}
\end{frame}

\begin{frame}
\frametitle{Implicaciones en estabilidad}
\begin{itemize}
  \item Un corte abrupto puede introducir ruido en la din\'amica.
\end{itemize}
\end{frame}

\subsection{Electrost\'atica de largo alcance}

\begin{frame}
\frametitle{Problema de Coulomb}
\begin{align*}
  u_{\text{C}}(r) &\propto \frac{1}{r} \quad \text{(largo alcance)}.
\end{align*}
\begin{itemize}
  \item El truncamiento directo produce grandes errores.
\end{itemize}
\end{frame}

\begin{frame}
\frametitle{Ewald}
\begin{align*}
  U &= U_{\text{real}} + U_{\text{rec}} + U_{\text{self}}.
\end{align*}
\begin{itemize}
  \item Divide Coulomb en suma real y rec\'iproca.
\end{itemize}
\end{frame}

\begin{frame}
\frametitle{PME}
\begin{itemize}
  \item Particle Mesh Ewald usa FFT para acelerar el c\'alculo.
\end{itemize}
\end{frame}

\begin{frame}
\frametitle{Precisi\'on}
\begin{itemize}
  \item Controla el error con par\'ametros de malla y \`alfa de Ewald.
\end{itemize}
\end{frame}

\begin{frame}
\frametitle{Elecci\'on pr\'actica}
\begin{itemize}
  \item Para biomol\'eculas en solvente, PME es el est\'andar.
\end{itemize}
\end{frame}

\subsection{Salida y reproducibilidad}

\begin{frame}
\frametitle{Trayectorias}
\begin{itemize}
  \item Guardar coordenadas y velocidades en intervalos regulares.
  \item Balance entre almacenamiento y resoluci\'on temporal.
\end{itemize}
\end{frame}

\begin{frame}
\frametitle{Observables}
\begin{align*}
  \langle A \rangle &\approx \frac{1}{M}\sum_{k=1}^M A_k.
\end{align*}
\begin{itemize}
  \item La autocorrelaci\'on determina el n\'umero efectivo de muestras.
\end{itemize}
\end{frame}

\begin{frame}
\frametitle{Semillas aleatorias}
\begin{itemize}
  \item Documentar seeds para reproducir termostatos estoc\'asticos.
\end{itemize}
\end{frame}

\begin{frame}
\frametitle{Checkpoints}
\begin{itemize}
  \item Guardar estados para reiniciar simulaciones largas.
\end{itemize}
\end{frame}

\begin{frame}
\frametitle{Resumen del episodio}
\begin{itemize}
  \item La integraci\'on num\'erica controla estabilidad y fidelidad.
  \item Termostatos y barostatos definen el ensemble.
  \item PBC y electrost\'atica de largo alcance son cr\'iticos en biomol\'eculas.
\end{itemize}
\end{frame}
