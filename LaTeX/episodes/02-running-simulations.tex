\section[Simulations]{Episode 2: Running simulations}

\subsection{Key parameters (OpenMM User Guide §3.7)}

\begin{frame}
\frametitle{Speed vs. accuracy trade-offs}
\begin{itemize}
  \item Speed: large time steps and reduced reporting.
  \item Numerical accuracy: symplectic integrators, constraints, and a small timestep.
  \item OpenMM exposes knobs for \texttt{dt}, \texttt{constraints}, \texttt{hydrogenMass}, and thermostats (§3.7.6–3.7.8).
\end{itemize}
\end{frame}

\begin{frame}
\frametitle{Time step and constraints}
\begin{align*}
  \Delta t &\approx \frac{1}{10\,\omega_{\text{max}}},\quad \omega_{\text{max}}\sim\sqrt{\frac{k}{\mu}}.
\end{align*}
\begin{itemize}
  \item With \texttt{constraints=HBonds}, the timestep doubles.
  \item \texttt{constraints=AllBonds/HAngles} allow bigger jumps but reduce flexibility.
\end{itemize}
\end{frame}

\begin{frame}
\frametitle{Heavy hydrogens}
\begin{align*}
  m_H^{\text{new}} &= \alpha\,m_H,\quad m_{heavy}^{\text{new}} = m_{heavy} - (\alpha-1)m_H.
\end{align*}
\begin{itemize}
  \item \texttt{hydrogenMass=1.5*amu} slows oscillations without changing the total mass.
  \item Used in integration scripts, e.g., \href{https://github.com/openmm/openmm/blob/master/examples/python-examples/argon-chemical-potential.py}{\texttt{argon-chemical-potential.py}} adopts this trick.
\end{itemize}
\end{frame}

\begin{frame}
\frametitle{Thermal and pressure coupling}
\begin{itemize}
  \item \texttt{LangevinIntegrator} simulates a heat bath with friction $\gamma$.
  \item Nosé-Hoover (chain) controls the canonical distribution while preserving momentum.
  \item Monte Carlo or Parrinello-Rahman barostats appear in the \href{https://github.com/openmm/openmm/blob/master/examples/python-examples/simulateAmber.py}{\texttt{simulateAmber.py}} scripts.
\end{itemize}
\end{frame}

\begin{frame}
\frametitle{Reporting and checkpoints}
\begin{itemize}
  \item `StateDataReporter` tracks energy, temperature, volume, and RMSD.
  \item `DCDReporter` / `NetCDFReporter` save trajectories (OpenMM App §3.13–3.14).
  \item Use `CheckpointReporter` to resume long simulations (OpenMM §3.15).
\end{itemize}
\end{frame}

\subsection{Time discretization}

\begin{frame}
\frametitle{Continuous and discrete equations}
\begin{align*}
  \dot{\mathbf{r}}(t) &= \mathbf{v}(t),\quad \dot{\mathbf{v}}(t) = \frac{\mathbf{F}(t)}{m}.
\end{align*}
\begin{itemize}
  \item Simulation replaces derivatives with finite increments.
\end{itemize}
\end{frame}

\begin{frame}
\frametitle{Time step}
\begin{align*}
  \mathbf{r}(t+\Delta t) &= \mathbf{r}(t) + \mathbf{v}(t)\Delta t + \mathcal{O}(\Delta t^2).
\end{align*}
\begin{itemize}
  \item $\Delta t$ must resolve the fastest vibrations.
\end{itemize}
\end{frame}

\begin{frame}
\frametitle{Local and global error}
\begin{align*}
  \text{error local} &= \mathcal{O}(\Delta t^{p+1}),\\
  \text{error global} &= \mathcal{O}(\Delta t^{p}).
\end{align*}
\begin{itemize}
  \item Order-$p$ integrators control the accumulation of error.
\end{itemize}
\end{frame}

\begin{frame}
\frametitle{Stability scale}
\begin{align*}
  \Delta t &\lesssim \frac{1}{10\,\omega_{\text{max}}}.
\end{align*}
\begin{itemize}
  \item $\omega_{\text{max}}$ comes from the fastest vibrational modes.
\end{itemize}
\end{frame}

\begin{frame}
\frametitle{Practical guideline}
\begin{itemize}
  \item Without constraints: $\Delta t \approx 1$ fs.
  \item With SHAKE/RATTLE: $\Delta t \approx 2$ fs.
\end{itemize}
\end{frame}

\subsection{Classical integrators}

\begin{frame}
\frametitle{Explicit Euler}
\begin{align*}
  \mathbf{v}_{t+\Delta t} &= \mathbf{v}_t + \frac{\mathbf{F}_t}{m}\Delta t,\\
  \mathbf{r}_{t+\Delta t} &= \mathbf{r}_t + \mathbf{v}_t\Delta t.
\end{align*}
\begin{itemize}
  \item Simple but unstable for molecular dynamics.
\end{itemize}
\end{frame}

\begin{frame}
\frametitle{Classical Verlet}
\begin{align*}
  \mathbf{r}_{t+\Delta t} &= 2\mathbf{r}_t - \mathbf{r}_{t-\Delta t} + \frac{\mathbf{F}_t}{m}\Delta t^2.
\end{align*}
\begin{itemize}
  \item Symmetric and time-reversible.
\end{itemize}
\end{frame}

\begin{frame}
\frametitle{Velocity Verlet}
\begin{align*}
  \mathbf{r}_{t+\Delta t} &= \mathbf{r}_t + \mathbf{v}_t\Delta t + \frac{\mathbf{F}_t}{2m}\Delta t^2,\\
  \mathbf{v}_{t+\Delta t} &= \mathbf{v}_t + \frac{\mathbf{F}_t+\mathbf{F}_{t+\Delta t}}{2m}\Delta t.
\end{align*}
\begin{itemize}
  \item Combines stability with direct access to velocities.
\end{itemize}
\end{frame}

\begin{frame}
\frametitle{Symmetry and stability}
\begin{itemize}
  \item Symplectic integrators conserve volume in phase space.
  \item They prevent energy drift in NVE.
\end{itemize}
\end{frame}

\begin{frame}
\frametitle{Practical choice}
\begin{itemize}
  \item OpenMM uses Verlet- or Langevin-type integrators.
  \item The choice depends on the desired ensemble.
\end{itemize}
\end{frame}

\subsection{Leapfrog scheme}

\begin{frame}
\frametitle{Leapfrog}
\begin{align*}
  \mathbf{v}_{t+\Delta t/2} &= \mathbf{v}_{t-\Delta t/2} + \frac{\mathbf{F}_t}{m}\Delta t,\\
  \mathbf{r}_{t+\Delta t} &= \mathbf{r}_t + \mathbf{v}_{t+\Delta t/2}\Delta t.
\end{align*}
\begin{itemize}
  \item Velocities and positions ``leapfrog'' through time.
\end{itemize}
\end{frame}

\begin{frame}
\frametitle{Graphical interpretation}
\begin{center}
  \includegraphics[width=0.72\linewidth]{leapfrog_method_argument.png}
\end{center}
\vspace{-0.1cm}
{\tiny Source: Wikimedia Commons (CC BY-SA 4.0). \cita{img-leapfrog}}
\end{frame}

\begin{frame}
\frametitle{Advantages}
\begin{itemize}
  \item Symmetric, reversible, and inexpensive.
  \item Preserves energy better than Euler.
\end{itemize}
\end{frame}

\begin{frame}
\frametitle{Relation to Verlet}
\begin{itemize}
  \item Leapfrog and velocity Verlet are algebraically equivalent.
  \item Differences appear in how velocities are stored.
\end{itemize}
\end{frame}

\begin{frame}
\frametitle{Integrator choice}
\begin{itemize}
  \item NVE: Verlet or Leapfrog.
  \item NVT: Langevin or Nos\'e-Hoover.
\end{itemize}
\end{frame}

\subsection{Conservation and NVE}

\begin{frame}
\frametitle{Total energy}
\begin{align*}
  E(t) &= K(t) + U(t).
\end{align*}
\begin{itemize}
  \item In ideal NVE, $E(t)$ is constant.
\end{itemize}
\end{frame}

\begin{frame}
\frametitle{Energy drift}
\begin{itemize}
  \item Numerical errors produce secular drift.
  \item Controlled by reducing $\Delta t$.
\end{itemize}
\end{frame}

\begin{frame}
\frametitle{Instantaneous temperature}
\begin{align*}
  T &= \frac{2\langle K \rangle}{3N\kb}.
\end{align*}
\begin{itemize}
  \item In NVE, $T$ fluctuates around a mean value.
\end{itemize}
\end{frame}

\begin{frame}
\frametitle{Quick checks}
\begin{itemize}
  \item Check stability of $E(t)$ and the velocity distribution.
  \item Compare $\langle K \rangle$ with the theoretical value.
\end{itemize}
\end{frame}

\begin{frame}
\frametitle{Equilibration time}
\begin{itemize}
  \item A pre-equilibration stabilizes the energy before collecting data.
\end{itemize}
\end{frame}

\subsection{Thermostats}

\begin{frame}
\frametitle{Thermostat goal}
\begin{itemize}
  \item Impose a canonical distribution at temperature $T$.
  \item Extract or inject energy in a controlled way.
\end{itemize}
\end{frame}

\begin{frame}
\frametitle{Langevin dynamics}
\begin{align*}
  m\ddot{\mathbf{r}} &= -\nabla U - \gamma m \dot{\mathbf{r}} + \mathbf{R}(t),\\
  \langle \mathbf{R}(t) \mathbf{R}(t') \rangle &= 2\gamma m\kb T\,\delta(t-t').
\end{align*}
\begin{itemize}
  \item Stochastic thermostat with friction and noise.
\end{itemize}
\end{frame}

\begin{frame}
\frametitle{Andersen}
\begin{itemize}
  \item Reassigns velocities randomly with frequency $\nu$.
  \item Good for sampling, less realistic dynamically.
\end{itemize}
\end{frame}

\begin{frame}
\frametitle{Nos\'e-Hoover}
\begin{align*}
  \dot{\mathbf{p}}_i &= \mathbf{F}_i - \xi\mathbf{p}_i,\quad \dot{\xi} = \frac{1}{Q}\left(\sum_i \frac{\mathbf{p}_i^2}{m_i} - 3N\kb T\right).
\end{align*}
\begin{itemize}
  \item Deterministic thermostat with an extended variable.
\end{itemize}
\end{frame}

\begin{frame}
\frametitle{Thermostat choice}
\begin{itemize}
  \item Langevin: robust and stable.
  \item Nos\'e-Hoover: more realistic dynamics if well tuned.
\end{itemize}
\end{frame}

\subsection{Barostats}

\begin{frame}
\frametitle{Microscopic pressure}
\begin{align*}
  P &= \frac{Nk_B T}{V} + \frac{1}{3V}\sum_{i<j}\mathbf{r}_{ij}\cdot\mathbf{F}_{ij}.
\end{align*}
\begin{itemize}
  \item Pressure depends on kinetic energy and internal forces.
\end{itemize}
\end{frame}

\begin{frame}
\frametitle{Berendsen barostat}
\begin{align*}
  \frac{dV}{dt} &= \frac{1}{\tau_P}(P_0-P)V.
\end{align*}
\begin{itemize}
  \item Fast to equilibrate, does not reproduce exact fluctuations.
\end{itemize}
\end{frame}

\begin{frame}
\frametitle{Parrinello-Rahman}
\begin{itemize}
  \item Scales the simulation cell with dynamic variables.
  \item Allows anisotropic volume changes.
\end{itemize}
\end{frame}

\begin{frame}
\frametitle{NPT}
\begin{itemize}
  \item Realistic ensemble for experimental conditions.
  \item Combines thermostat and barostat.
\end{itemize}
\end{frame}

\begin{frame}
\frametitle{Practical choice}
\begin{itemize}
  \item Equilibration: Berendsen + Langevin.
  \item Production: more rigorous barostat (Monte Carlo barostat).
\end{itemize}
\end{frame}

\subsection{Constraints}

\begin{frame}
\frametitle{Motivation}
\begin{itemize}
  \item Eliminating fast vibrations allows increasing $\Delta t$.
\end{itemize}
\end{frame}

\begin{frame}
\frametitle{Holonomic constraints}
\begin{align*}
  g_k(\mathbf{r}) &= 0,\quad k=1,\dots,M.
\end{align*}
\begin{itemize}
  \item Fix distances or internal angles.
\end{itemize}
\end{frame}

\begin{frame}
\frametitle{SHAKE/RATTLE}
\begin{itemize}
  \item SHAKE corrects positions; RATTLE corrects positions and velocities.
\end{itemize}
\end{frame}

\begin{frame}
\frametitle{Numerical impact}
\begin{itemize}
  \item Increases stability and reduces simulation cost.
  \item May alter high-frequency modes.
\end{itemize}
\end{frame}

\begin{frame}
\frametitle{Recommendation}
\begin{itemize}
  \item Use constraints on H bonds for biomolecules.
\end{itemize}
\end{frame}

\subsection{Periodic boundary conditions}

\begin{frame}
\frametitle{Motivation for PBC}
\begin{itemize}
  \item Avoids surface effects in finite systems.
  \item Mimics an infinite system by replicating the cell.
\end{itemize}
\end{frame}

\begin{frame}
\frametitle{Periodic cell}
\begin{align*}
  \mathbf{r} &\to \mathbf{r} + n_x\mathbf{a} + n_y\mathbf{b} + n_z\mathbf{c}.
\end{align*}
\begin{itemize}
  \item $\mathbf{a},\mathbf{b},\mathbf{c}$ define the simulation box.
\end{itemize}
\end{frame}

\begin{frame}
\frametitle{2D example}
\begin{center}
  \includegraphics[width=0.72\linewidth]{pbc_2d.png}
\end{center}
\vspace{-0.1cm}
{\tiny Source: Wikimedia Commons (CC BY-SA 4.0). \cita{img-pbc-2d}}
\end{frame}

\begin{frame}
\frametitle{Implications for dynamics}
\begin{itemize}
  \item Particles leaving re-enter through the opposite face.
  \item The density in the simulated volume stays constant.
\end{itemize}
\end{frame}

\begin{frame}
\frametitle{Non-orthogonal boxes}
\begin{itemize}
  \item Triclinic cells for crystals or tilted membranes.
\end{itemize}
\end{frame}

\subsection{Minimum image}

\begin{frame}
\frametitle{Minimum image convention}
\begin{align*}
  r_{ij} &= \min_{\mathbf{n}} \lVert \mathbf{r}_i-\mathbf{r}_j+\mathbf{n} \rVert.
\end{align*}
\begin{itemize}
  \item Use the nearest periodic image to compute interactions.
\end{itemize}
\end{frame}

\begin{frame}
\frametitle{Visual example}
\begin{center}
  \includegraphics[width=0.7\linewidth]{minimum_image_convention.png}
\end{center}
\vspace{-0.1cm}
{\tiny Source: Wikimedia Commons (CC BY-SA 4.0). \cita{img-min-image}}
\end{frame}

\begin{frame}
\frametitle{Cutoff radius}
\begin{itemize}
  \item The minimum image requires $r_c < L/2$ to avoid double counting.
\end{itemize}
\end{frame}

\begin{frame}
\frametitle{Energy impact}
\begin{itemize}
  \item Errors arise if the cutoff is large or the density is low.
\end{itemize}
\end{frame}

\begin{frame}
\frametitle{Best practices}
\begin{itemize}
  \item Adjust $r_c$ and box size according to the system density.
\end{itemize}
\end{frame}

\subsection{Cutoffs and smoothing}

\begin{frame}
\frametitle{Short-range interactions}
\begin{itemize}
  \item LJ is truncated at $r_c$ with or without a smoothing function.
\end{itemize}
\end{frame}

\begin{frame}
\frametitle{Switch function}
\begin{align*}
  u_{\text{switch}}(r) &= s(r)\,u(r),\quad s(r)\in[0,1].
\end{align*}
\begin{itemize}
  \item Avoids discontinuities in forces.
\end{itemize}
\end{frame}

\begin{frame}
\frametitle{Tail correction}
\begin{align*}
  U_{\text{tail}} &\approx 2\pi\rho \int_{r_c}^{\infty} r^2 u(r)\,dr.
\end{align*}
\begin{itemize}
  \item Corrects energy and pressure when truncating LJ.
\end{itemize}
\end{frame}

\begin{frame}
\frametitle{Choosing $r_c$}
\begin{itemize}
  \item Larger $r_c$ improves accuracy but increases cost.
\end{itemize}
\end{frame}

\begin{frame}
\frametitle{Stability implications}
\begin{itemize}
  \item An abrupt cutoff can introduce noise in the dynamics.
\end{itemize}
\end{frame}

\subsection{Long-range electrostatics}

\begin{frame}
\frametitle{Coulomb problem}
\begin{align*}
  u_{\text{C}}(r) &\propto \frac{1}{r} \quad \text{(long range)}.
\end{align*}
\begin{itemize}
  \item Direct truncation produces large errors.
\end{itemize}
\end{frame}

\begin{frame}
\frametitle{Ewald}
\begin{align*}
  U &= U_{\text{real}} + U_{\text{rec}} + U_{\text{self}}.
\end{align*}
\begin{itemize}
  \item Split Coulomb into real and reciprocal sums.
\end{itemize}
\end{frame}

\begin{frame}
\frametitle{PME}
\begin{itemize}
  \item Particle Mesh Ewald uses FFT to speed up the calculation.
\end{itemize}
\end{frame}

\begin{frame}
\frametitle{Accuracy}
\begin{itemize}
  \item Control the error with mesh parameters and the Ewald alpha.
\end{itemize}
\end{frame}

\begin{frame}
\frametitle{Practical choice}
\begin{itemize}
  \item For solvated biomolecules, PME is the standard.
\end{itemize}
\end{frame}

\subsection{Outputs and reproducibility}

\begin{frame}
\frametitle{Trajectories}
\begin{itemize}
  \item Save coordinates and velocities at regular intervals.
  \item Balance between storage and temporal resolution.
\end{itemize}
\end{frame}

\begin{frame}
\frametitle{Observables}
\begin{align*}
  \langle A \rangle &\approx \frac{1}{M}\sum_{k=1}^M A_k.
\end{align*}
\begin{itemize}
  \item Autocorrelation determines the effective number of samples.
\end{itemize}
\end{frame}

\begin{frame}
\frametitle{Random seeds}
\begin{itemize}
  \item Document seeds to reproduce stochastic thermostats.
\end{itemize}
\end{frame}

\begin{frame}
\frametitle{Checkpoints}
\begin{itemize}
  \item Save states to restart long simulations.
\end{itemize}
\end{frame}

\begin{frame}
\frametitle{Episode summary}
\begin{itemize}
  \item Numerical integration controls stability and accuracy.
  \item Thermostats and barostats set the ensemble.
  \item PBC and long-range electrostatics are critical in biomolecules.
\end{itemize}
\end{frame}
