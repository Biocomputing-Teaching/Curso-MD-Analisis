\section[PyEMMA]{Episode 6: Markov models with PyEMMA}

\subsection{Motivation and theoretical background}

\begin{frame}
\frametitle{Why Markov State Models?}
\begin{itemize}
  \item Molecular dynamics (MD) simulations generate high-dimensional trajectories
  $\{\mathbf{X}_t\}_{t=0}^T$.
  \item Relevant molecular processes occur on timescales much longer than MD timesteps.
  \item MSMs provide a statistical coarse-graining into discrete states with Markovian dynamics.
  \item Enable computation of long-timescale kinetics, populations, and pathways.
\end{itemize}
\end{frame}

\begin{frame}
\frametitle{From continuous dynamics to a Markov chain}
\begin{itemize}
  \item Consider a stochastic process $\mathbf{X}_t$ in phase space $\Omega$.
  \item Partition $\Omega$ into disjoint sets $\{S_1,\dots,S_N\}$.
  \item Define a discrete process $X_t \in \{1,\dots,N\}$:
  \[
    X_t = i \quad \text{if } \mathbf{X}_t \in S_i.
  \]
  \item Markov assumption at lag time $\tau$:
  \[
    P(X_{t+\tau}=j \mid X_t=i, \ldots)
    \approx
    P(X_{t+\tau}=j \mid X_t=i).
  \]
\end{itemize}
\end{frame}

%====================================================
\subsection{Trajectory preparation}

\begin{frame}
\frametitle{Preparing molecular trajectories}
\begin{itemize}
  \item Input trajectories from MD engines (OpenMM, Gromacs, AMBER, \dots).
  \item Preprocessing:
  \begin{itemize}
    \item Remove periodic boundary artifacts.
    \item Align structures to a reference.
    \item Remove solvent if not used as features.
    \item Subsample to a uniform timestep $\Delta t$.
  \end{itemize}
  \item Validate trajectories: energy stability, RMSD convergence.
\end{itemize}
\end{frame}

%====================================================
\subsection{Feature representation}

\begin{frame}
\frametitle{Feature extraction}
\begin{itemize}
  \item Each frame is mapped to a feature vector $\mathbf{x}_t \in \mathbb{R}^d$.
  \item Typical features:
  \begin{itemize}
    \item Interatomic distances or contacts.
    \item Dihedral angles.
    \item Ligand--protein distances.
  \end{itemize}
  \item Features should resolve slow collective motions.
\end{itemize}
\end{frame}

\begin{frame}
\frametitle{Covariance structure}
\begin{align*}
  \bar{\mathbf{x}} &= \langle \mathbf{x}_t \rangle, \\
  C_0 &= \langle (\mathbf{x}_t-\bar{\mathbf{x}})
          (\mathbf{x}_t-\bar{\mathbf{x}})^T \rangle.
\end{align*}
\begin{itemize}
  \item Averages over all frames and trajectories.
  \item $C_0$ captures instantaneous correlations.
\end{itemize}
\end{frame}

%====================================================
\subsection{Time-lagged Independent Component Analysis}

\begin{frame}
\frametitle{Time-lagged covariance}
\begin{align*}
  C_\tau =
  \langle (\mathbf{x}_t-\bar{\mathbf{x}})
  (\mathbf{x}_{t+\tau}-\bar{\mathbf{x}})^T \rangle.
\end{align*}
\begin{itemize}
  \item Measures correlations persisting over lag time $\tau$.
  \item Slow processes correspond to large time-lagged correlations.
  \item This procedure is known as time-lagged independent component analysis (TICA): the eigenvectors of $C_\tau$ define the slow collective coordinates (tICs).
\end{itemize}
\end{frame}

\begin{frame}
\frametitle{Generalized eigenvalue problem}
\begin{align*}
  C_\tau \mathbf{w}_i = \lambda_i C_0 \mathbf{w}_i.
\end{align*}
\begin{itemize}
  \item Eigenvectors $\mathbf{w}_i$ define tICs.
  \item Projection:
  \[
    y_{i,t} = \mathbf{w}_i^T \mathbf{x}_t.
  \]
\end{itemize}
\end{frame}

\begin{frame}
\frametitle{Implied timescales}
\begin{align*}
  t_i = -\frac{\tau}{\ln \lambda_i}.
\end{align*}
\begin{itemize}
  \item Estimates relaxation timescales of slow modes.
  \item Plateaus vs.\ $\tau$ indicate robust dynamics.
\end{itemize}
\end{frame}

%====================================================
\subsection{Discretization}

\begin{frame}
\frametitle{Clustering into microstates}
\begin{itemize}
  \item Project data onto first $m$ tICs.
  \item Cluster in reduced space (e.g.\ $k$-means).
  \item Each frame assigned to a discrete state $i$.
\end{itemize}
\end{frame}

\begin{frame}
\frametitle{Discrete trajectories}
\begin{itemize}
  \item Continuous trajectories become symbol sequences:
  \[
    X^{(n)} = (x^{(n)}_0, \dots, x^{(n)}_{T_n}).
  \]
  \item These sequences define the MSM input.
\end{itemize}
\end{frame}

%====================================================
\subsection{MSM estimation}

\begin{frame}
\frametitle{Transition counts}
\begin{align*}
  C_{ij}(\tau) =
  \sum_t \mathbb{I}(X_t=i, X_{t+\tau}=j).
\end{align*}
\begin{itemize}
  \item Counts transitions from $i$ to $j$ at lag time $\tau$.
\end{itemize}
\end{frame}

\begin{frame}
\frametitle{Transition matrix}
\begin{align*}
  T_{ij}(\tau) =
  \frac{C_{ij}(\tau)}{\sum_k C_{ik}(\tau)}.
\end{align*}
\begin{itemize}
  \item Row-stochastic matrix.
  \item Interpreted as conditional probabilities.
\end{itemize}
\end{frame}

\begin{frame}
\frametitle{Detailed balance}
\begin{align*}
  \pi_i T_{ij} = \pi_j T_{ji}.
\end{align*}
\begin{itemize}
  \item Expected for equilibrium simulations.
  \item Enforcing reversibility reduces statistical noise.
\end{itemize}
\end{frame}

%====================================================
\subsection{Spectral analysis and validation}

\begin{frame}
\frametitle{Stationary distribution}
\begin{align*}
  \boldsymbol{\pi}^T T = \boldsymbol{\pi}^T.
\end{align*}
\begin{itemize}
  \item $\pi_i$ gives equilibrium populations.
  \item Free energies:
  \[
    F_i = -k_B T \ln \pi_i + \text{const}.
  \]
\end{itemize}
\end{frame}

\begin{frame}
\frametitle{Eigenvalues and timescales}
\begin{align*}
  T \mathbf{r}_i = \lambda_i \mathbf{r}_i,
  \qquad
  t_i = -\frac{\tau}{\ln \lambda_i}.
\end{align*}
\begin{itemize}
  \item $\lambda_1 = 1$ corresponds to equilibrium.
  \item Spectral gap indicates timescale separation.
\end{itemize}
\end{frame}

\begin{frame}
\frametitle{Chapman--Kolmogorov test}
\begin{itemize}
  \item Markovianity check:
  \[
    T(n\tau) \approx T(\tau)^n.
  \]
  \item Agreement validates chosen lag time.
  \item Passing CK ensures that the implied kinetics remain invariant when propagation is computed at multiples of the base lag, confirming the MSM describes the same slow modes.
\end{itemize}
\end{frame}

%====================================================
\begin{frame}
\frametitle{Transition Path Theory overview}
\begin{itemize}
  \item TPT builds on the MSM coarse graining to identify dominant pathways between user-defined reactant and product macrostates.
  \item It solves for the committor (probability of reaching the product before returning to the reactant) and computes reactive fluxes that quantify the net probability current supporting those transitions.
  \item Together the committor and flux highlight where the slow dynamics concentrate and which state-to-state hops carry the most weight in the MSM.
\end{itemize}
\end{frame}

%====================================================
\subsection{Metastable coarse graining with PCCA++}

\begin{frame}
\frametitle{PCCA++ metastable clustering}
\begin{itemize}
  \item Perron cluster cluster analysis (PCCA++) exploits the slow eigenvectors of the MSM transition matrix to define fuzzy memberships to macrostates.
  \item Each microstate carries a vector of probabilities, leading to metastable sets that preserve the kinetics encoded in the slow modes.
  \item Coarse-grained states are therefore suited for human interpretation and downstream analysis (MFPTs, representative structures, etc.).
  \item Selecting a small number of macrostates keeps the essential long-timescale behavior and prepares reactant/product sets for TPT.
\end{itemize}
\end{frame}

\begin{frame}
\frametitle{Metastable set assignments}
\begin{itemize}
  \item A crisp assignment can be obtained by taking the argmax of the membership vector for each microstate.
  \item The resulting macrostates label dense basins or transition regions in the slow collective coordinates.
  \item These macrostates control the initial/final sets in TPT computations, ensuring that committors and fluxes are defined between physically meaningful ensembles.
\end{itemize}
\end{frame}

%====================================================
\subsection{Transition Path Theory}

\begin{frame}
\frametitle{Committor function}
\begin{itemize}
  \item Define reactant set $A$ and product set $B$.
  \item Forward committor:
  \[
    q_i = \sum_j T_{ij} q_j,
    \quad
    q_i=0 \ (i\in A),\;
    q_i=1 \ (i\in B).
  \]
  \item The committor is the probability to reach $B$ before returning to $A$ and defines reactive surfaces.
\end{itemize}
\end{frame}

\begin{frame}
\frametitle{Reactive fluxes}
\begin{itemize}
  \item TPT flux:
  \[
    f_{ij} = \pi_i T_{ij} q_i (1-q_j)
  \]
  \item Measures net current along transition tubes and highlights dominant pathways.
\end{itemize}
\end{frame}

\begin{frame}
\frametitle{Reactive flux}
\begin{align*}
  f_{ij} = \pi_i T_{ij} q_i (1-q_j).
\end{align*}
\begin{itemize}
  \item Identifies dominant reactive pathways.
\end{itemize}
\end{frame}
