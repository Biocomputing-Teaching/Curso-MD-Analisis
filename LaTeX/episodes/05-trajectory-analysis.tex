\section[Análisis]{Episodio 5: Análisis de trayectorias}

\subsection{Preprocesado y limpieza}

\begin{frame}
\frametitle{Pipeline previo al análisis}
\begin{enumerate}
  \item Convertir y limpiar trayectorias con `simulatePdb.py` o `simulateGromacs.py`.
  \item Alinear estructuras y eliminar solvente irrelevante.
  \item Definir subconjuntos de átomos y escribir DCD/NetCDF para análisis posteriores.
\end{enumerate}
{\tiny Scripts del Application Layer de OpenMM. \cita{OpenMMUserguide2024}}
\end{frame}

\begin{frame}
\frametitle{Velocidades y distribuciones}
\begin{itemize}
  \item `StateDataReporter` y `DCDReporter` construyen histogramas de $v_i$ y energías.
  \item Por el teorema de equipartición: $\langle K \rangle = \frac{3N}{2}\kbt$ para sistemas clásicos.
  \item Controlar $C(t)=\langle A(0)A(t)\rangle$ arroja las escalas relevantes.
\end{itemize}
\end{frame}

\begin{frame}
\frametitle{RMSD y alineamientos}
\begin{align*}
  \text{RMSD}(t) &= \sqrt{\frac{1}{N}\sum_{i=1}^N\lVert\mathbf{r}_i(t)-\mathbf{r}_i^{\text{ref}}\rVert^2}.
\end{align*}
\begin{itemize}
  \item Usamos el algoritmo de Kabsch para rotaciones óptimas.
  \item Documentamos con `simulateAmber.py` y la salida en CSV para cada réplica.
\end{itemize}
\end{frame}

\subsection{Velocidades e inercia}

\begin{frame}
\frametitle{Radio de giro y tensor de inercia}
\begin{align*}
  I_{ab} &= \sum_i m_i\left(r_i^2\delta_{ab}-r_{ia}r_{ib}\right).
\end{align*}
\begin{itemize}
  \item Autovalores de $I$ describen la anisotropía global.
  \item Monitorear $R_g^2=\frac{1}{N}\sum_i m_i\lVert \mathbf{r}_i-\mathbf{r}_{\text{CM}}\rVert^2$ por réplica.
\end{itemize}
\end{frame}

\begin{frame}
\frametitle{Extremos y contactos}
\begin{itemize}
  \item Mapas binarios con umbral de 0.45 nm generan matrices de contacto.
  \item Las columnas representan cambios de dominios detectados por `simulateCharmm.py`.
\end{itemize}
\end{frame}

\subsection{Reducción y clustering}

\begin{frame}
\frametitle{Covarianza y PCA}
\begin{align*}
  C &= \langle (\mathbf{x}-\langle \mathbf{x}\rangle)(\mathbf{x}-\langle \mathbf{x}\rangle)^T\rangle.
\end{align*}
\begin{itemize}
  \item Proyectamos sobre los primeros autovectores y definimos ejes de evento.
  \item Relacionamos con \texttt{coarse\_grained\_polymer.py} para comparar modos lentos de polímeros y proteínas.
\end{itemize}
\end{frame}

\begin{frame}
\frametitle{Agrupamiento jerárquico}
\begin{itemize}
  \item Usamos enlaces promedio e índice silhouette para decidir $k$.
  \item Una mala elección produce microestados vacíos; preferimos ensayos progresivos.
\end{itemize}
\end{frame}

\begin{frame}
\frametitle{Transiciones y grafos}
\begin{itemize}
  \item Grafos de estados (ver `graph.png`) representan canales cinéticos.
  \item Los bordes anotados reflejan flujos medidos por scripts de OpenMMTools.
\end{itemize}
\end{frame}

\begin{frame}
\frametitle{Autovalores y modos}
\begin{align*}
  C\mathbf{w} &= \lambda \mathbf{w},\quad \lambda \in \mathbb{R}_+.
\end{align*}
\begin{itemize}
  \item Los modos dominantes identifican movimientos colectivos.
  \item Guardamos proyecciones para alimentar modelos de Markov.
\end{itemize}
\end{frame}

\subsection{Energías libres y superficies}

\begin{frame}
\frametitle{Funciones de densidad}
\begin{align*}
  g(r) &= \frac{1}{4\pi r^2\rho}\left\langle \sum_{i\ne j}\delta(r-r_{ij})\right\rangle.
\end{align*}
\begin{itemize}
  \item Comparar con `argon-chemical-potential.py` para validar.
  \item Usa histogramas reponderados para estimar $F(r)=-\kbt\ln g(r)$.
\end{itemize}
\end{frame}

\begin{frame}
\frametitle{Superficies de energía libre y barreras}
\begin{itemize}
  \item La proyección $F(s)=-\kbt\ln P(s)$ requiere CVs robustos.
  \item Integrar con ventanas de `umbrella sampling` y combinar con MBAR.
  \item Comparar perfiles unidimensionales con `simulateAmber.py`.
  \item Validar con histogramas solapados manualmente.
\end{itemize}
\end{frame}

\subsection{Estadística y confianza}

\begin{frame}
\frametitle{Bootstrap y bloqueos}
\begin{itemize}
  \item Re-muestrear bloques para estimar varianzas.
  \item Las autocorrelaciones (ver \texttt{bs\_autocorrelation.pdf}) determinan el tamaño del bloque.
\end{itemize}
\end{frame}

\begin{frame}
\frametitle{Autocorrelaciones de bloques}
\begin{center}
  \includegraphics[width=0.78\linewidth]{episodes/bs_autocorrelation.pdf}
\end{center}
{\tiny Fuente: OpenMM Cookbook (simulation parameters). \cita{img-bs-autocorrelation}}
\end{frame}

\begin{frame}
\frametitle{Serie de errores}
\begin{align*}
  \sigma_{\bar{A}} &= \frac{\sigma_A}{\sqrt{N_{\text{eff}}}},\quad N_{\text{eff}}=\frac{N}{g},\quad g=1+2\sum_t\rho(t).
\end{align*}
\end{frame}

\begin{frame}
\frametitle{Reportes finales}
\begin{itemize}
  \item Generar tablas de RMSD/RMSF/radio de giro con `simulateCharmm.py`.
  \item Exportar CSV/HDF para facilitar el paso a PyEMMA/Deeptime.
\end{itemize}
\end{frame}

\subsection{Resumen}

\begin{frame}
\frametitle{Resumen y siguiente paso}
\begin{itemize}
  \item Preprocesado sin PBC artefactual.
  \item CVs estables y histogramas convergentes.
  \item Reportes con errores y trayectorias documentadas.
  \item Usar los features preparados como entrada para PyEMMA/Deeptime.
\end{itemize}
\end{frame}
