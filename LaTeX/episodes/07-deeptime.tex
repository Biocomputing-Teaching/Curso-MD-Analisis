\section[Deeptime]{Episode 7: Models and spectra with Deeptime}

\subsection{Transfer operators}

\begin{frame}
\frametitle{Definition}
\begin{align*}
  (\mathcal{T}_\tau f)(x) &= \mathbb{E}[f(X_{t+\tau})\mid X_t=x].
\end{align*}
\begin{itemize}
  \item Extends MSM to the continuous space of observables.
  \item PyEMMA and Deeptime share the dataset generated by \href{https://github.com/openmm/openmm/blob/master/examples/python-examples/simulateAmber.py}{\texttt{simulateAmber.py}}.
\end{itemize}
\end{frame}

\begin{frame}
\frametitle{Transfer-operator diagram}
\begin{center}
  \includegraphics[width=0.75\linewidth]{transfer_operator.png}
\end{center}
{\tiny Visual summary of the operators linking observed coordinates to future distributions.}
\end{frame}

\begin{frame}
\frametitle{Koopman vs Perron-Frobenius}
\begin{itemize}
  \item Koopman acts on observables (functions).
  \item Perron-Frobenius acts on densities.
  \item They enable building the `TransferOperator` in Deeptime.
\end{itemize}
\end{frame}

\subsection{Spectrum and modes}

\begin{frame}
\frametitle{Spectral problem}
\begin{align*}
  \mathcal{T}_\tau \psi_i &= \lambda_i \psi_i.
\end{align*}
\begin{itemize}
  \item The eigenvalues \(\lambda_i\) define implied times \(t_i=-\tau/\ln\lambda_i\).
  \item \(\lambda_1=1\) for equilibrium and values close to 1 signal slow processes.
\end{itemize}
\end{frame}

\begin{frame}
\frametitle{Modos espectrales}
\begin{center}
  \includegraphics[width=0.72\linewidth]{eigenvects.pdf}
\end{center}
{\tiny Visualization of slow modes (OpenMM Cookbook REST). \cita{img-eigenvects}}
\end{frame}

\subsection{Bases and projections}

\begin{frame}
\frametitle{Basis expansion}
\begin{align*}
  f(x) &= \sum_i c_i \phi_i(x).
\end{align*}
\begin{itemize}
  \item Deeptime allows linear/nonlinear bases: kernels and networks.
\end{itemize}
\end{frame}

\begin{frame}
\frametitle{Bases y proyecciones}
\begin{columns}
  \column{0.5\linewidth}
  \begin{center}
    \includegraphics[width=\linewidth]{Linalg_non_orthog_basis_R2_2-1.png}
  \end{center}
  \column{0.5\linewidth}
  \begin{center}
    \includegraphics[width=\linewidth]{Linalg_xyperp_not_yz.png}
  \end{center}
\end{columns}
{\tiny Source: reference pending. \cita{unknown}}
\end{frame}

\begin{frame}
\frametitle{Error and regularization}
\begin{itemize}
  \item Poor bases introduce spectral bias.
  \item Regularize with \(L_2\) and cross-validation.
\end{itemize}
\end{frame}

\subsection{tICA and VAMP}

\begin{frame}
\frametitle{tICA in Deeptime}
\begin{align*}
  C_\tau \mathbf{w} &= \lambda C_0 \mathbf{w}.
\end{align*}
\begin{itemize}
  \item Maximizes long-term correlation.
\end{itemize}
\end{frame}

\begin{frame}
\frametitle{VAMP score}
\begin{align*}
  \mathcal{R}_2 &= \sum_i \sigma_i^2.
\end{align*}
\begin{itemize}
  \item Evaluates spectral quality to select features.
\end{itemize}
\end{frame}

\subsection{Advanced models}

\begin{frame}
\frametitle{Koopman approximation}
\begin{itemize}
  \item Expand observables in linear/nonlinear bases.
  \item Obtain reduced, more interpretable representations.
\end{itemize}
\end{frame}

\begin{frame}
\frametitle{Kernel and neural models}
\begin{itemize}
  \item Kernels capture nonlinearities; networks demand strict validation.
  \item Regularize with dropout or \(L_2\) to prevent overfitting.
\end{itemize}
\end{frame}

\subsection{Validation and scores}

\begin{frame}
\frametitle{Validation and robustness}
\begin{itemize}
  \item Splitting into time blocks prevents information leakage.
  \item Compare spectral scores across ensembles.
  \item \(T(n\tau)\approx T(\tau)^n\) tests temporal consistency.
  \item Vary \(\tau\) and clustering to detect sensitivity.
\end{itemize}
\end{frame}

\subsection{Pipeline with Deeptime}

\begin{frame}
\frametitle{Basic workflow}
\begin{itemize}
  \item Features \(\rightarrow\) tICA/VAMP \(\rightarrow\) spectral model.
  \item Use \href{https://github.com/openmm/openmm/blob/master/examples/python-examples/simulatePdb.py}{\texttt{simulatePdb.py}} to generate trajectories and `OpenMMTools` to prepare `REST`.
\end{itemize}
\end{frame}

\begin{frame}
\frametitle{Visualization}
\begin{itemize}
  \item 2D projections with FES and comparison with MSM.
\end{itemize}
\end{frame}

\subsection{Summary}

\begin{frame}
\frametitle{Episode summary}
\begin{itemize}
  \item Deeptime extends MSM with spectral operators and cross-validation.
  \item Combine features, VAMP, and scores to select robust models.
\end{itemize}
\end{frame}
