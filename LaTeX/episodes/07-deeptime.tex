\section[Deeptime]{Episodio 7: Modelos y espectros con Deeptime}

\subsection{Operadores de transferencia}

\begin{frame}
\frametitle{Definición}
\begin{align*}
  (\mathcal{T}_\tau f)(x) &= \mathbb{E}[f(X_{t+\tau})\mid X_t=x].
\end{align*}
\begin{itemize}
  \item Extiende MSM al espacio continuo de observables.
  \item PyEMMA y Deeptime comparten la base de datos generada por \href{https://github.com/openmm/openmm/blob/master/examples/python-examples/simulateAmber.py}{\texttt{simulateAmber.py}}.
\end{itemize}
\end{frame}

\begin{frame}
\frametitle{Koopman vs Perron-Frobenius}
\begin{itemize}
  \item Koopman actúa sobre observables (funciones).
  \item Perron-Frobenius actúa sobre densidades.
  \item Permiten construir `TransferOperator` en Deeptime.
\end{itemize}
\end{frame}

\subsection{Espectro y modos}

\begin{frame}
\frametitle{Problema espectral}
\begin{align*}
  \mathcal{T}_\tau \psi_i &= \lambda_i \psi_i.
\end{align*}
\begin{itemize}
  \item Los autovalores \(\lambda_i\) definen tiempos implícitos \(t_i=-\tau/\ln\lambda_i\).
  \item \(\lambda_1=1\) para equilibrio y valores próximos a 1 indican procesos lentos.
\end{itemize}
\end{frame}

\begin{frame}
\frametitle{Modos espectrales}
\begin{center}
  \includegraphics[width=0.72\linewidth]{episodes/eigenvects.pdf}
\end{center}
{\tiny Visualización de modos lentos (OpenMM Cookbook REST). \cita{img-eigenvects}}
\end{frame}

\subsection{Bases y proyecciones}

\begin{frame}
\frametitle{Expansión en bases}
\begin{align*}
  f(x) &= \sum_i c_i \phi_i(x).
\end{align*}
\begin{itemize}
  \item Deeptime permite bases lineales/no lineales: kernels y redes.
\end{itemize}
\end{frame}

\begin{frame}
\frametitle{Bases y proyecciones}
\begin{columns}
  \column{0.5\linewidth}
  \begin{center}
    \includegraphics[width=\linewidth]{episodes/Linalg_non_orthog_basis_R2_2-1.png}
  \end{center}
  \column{0.5\linewidth}
  \begin{center}
    \includegraphics[width=\linewidth]{episodes/Linalg_xyperp_not_yz.png}
  \end{center}
\end{columns}
{\tiny Fuente: referencia pendiente. \cita{unknown}}
\end{frame}

\begin{frame}
\frametitle{Error y regularización}
\begin{itemize}
  \item Bases pobres generan sesgo espectral.
  \item Regularizar con \(L_2\) y validación cruzada.
\end{itemize}
\end{frame}

\subsection{tICA y VAMP}

\begin{frame}
\frametitle{tICA en Deeptime}
\begin{align*}
  C_\tau \mathbf{w} &= \lambda C_0 \mathbf{w}.
\end{align*}
\begin{itemize}
  \item Maximiza correlación a largo plazo.
\end{itemize}
\end{frame}

\begin{frame}
\frametitle{VAMP score}
\begin{align*}
  \mathcal{R}_2 &= \sum_i \sigma_i^2.
\end{align*}
\begin{itemize}
  \item Evalúa calidad espectral para escoger features.
\end{itemize}
\end{frame}

\subsection{Modelos avanzados}

\begin{frame}
\frametitle{Aproximación de Koopman}
\begin{itemize}
  \item Expandir observables en bases lineales/no lineales.
  \item Obtenemos representaciones reducidas y más interpretables.
\end{itemize}
\end{frame}

\begin{frame}
\frametitle{Modelos kernel y neuronales}
\begin{itemize}
  \item Kernels capturan no linealidades; redes requieren validación estricta.
  \item Regularizar con dropout o \(L_2\) para prevenir sobreajuste.
\end{itemize}
\end{frame}

\subsection{Validación y scores}

\begin{frame}
\frametitle{Validación y robustez}
\begin{itemize}
  \item Dividir en bloques temporales evita fuga de información.
  \item Comparar scores espectrales entre conjuntos.
  \item \(T(n\tau)\approx T(\tau)^n\) valida consistencia temporal.
  \item Variar \(\tau\) y clustering permite detectar sensibilidad.
\end{itemize}
\end{frame}

\subsection{Pipeline con Deeptime}

\begin{frame}
\frametitle{Flujo básico}
\begin{itemize}
  \item Features \(\rightarrow\) tICA/VAMP \(\rightarrow\) modelo espectral.
  \item Usar \href{https://github.com/openmm/openmm/blob/master/examples/python-examples/simulatePdb.py}{\texttt{simulatePdb.py}} para generar trayectorias y `OpenMMTools` para preparar `REST`.
\end{itemize}
\end{frame}

\begin{frame}
\frametitle{Visualización}
\begin{itemize}
  \item Proyecciones 2D con FES y comparación con MSM.
\end{itemize}
\end{frame}

\subsection{Resumen}

\begin{frame}
\frametitle{Resumen del episodio}
\begin{itemize}
  \item Deeptime extiende MSM con operadores espectrales y validación cruzada.
  \item Combinamos features, VAMP y scores para seleccionar modelos robustos.
\end{itemize}
\end{frame}
